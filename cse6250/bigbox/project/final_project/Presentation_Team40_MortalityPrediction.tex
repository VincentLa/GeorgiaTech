\documentclass{beamer}
\def\Put(#1,#2)#3{\leavevmode\makebox(0,0){\put(#1,#2){#3}}}
\usepackage{tabularx, booktabs}
\usetheme{Frankfurt}
\usepackage{caption}
\usepackage{color}
\usepackage{epsfig}
\usepackage{epstopdf}
\usepackage{pdfpages}


\captionsetup[table]{labelformat = empty}
\setbeamertemplate{navigation symbols}{} % Suppress the navigation symbols that normally appear on the bottom of Beamer slides

% Redefine the footer to three-panel split with Author | Title | Slide number instead of Warsaw default
\makeatletter
\setbeamertemplate{footline}
{
  \leavevmode%
  \hbox{%
  \begin{beamercolorbox}[wd=.25\paperwidth,ht=2.25ex,dp=1ex,center]{author in head/foot}%
    \usebeamerfont{author in head/foot}\insertshortauthor%~~\beamer@ifempty{\insertshortinstitute}{}{(\insertshortinstitute)}
  \end{beamercolorbox}%
  \begin{beamercolorbox}[wd=.5\paperwidth,ht=2.25ex,dp=1ex,center]{title in head/foot}%
    \usebeamerfont{title in head/foot}\insertshorttitle
  \end{beamercolorbox}%
  \begin{beamercolorbox}[wd=.25\paperwidth,ht=2.25ex,dp=1ex,right]{date in head/foot}%
   % \usebeamerfont{date in head/foot}\insertshortdate{}\hspace*{2em}
    \insertframenumber{} / \inserttotalframenumber\hspace*{2ex} 
  \end{beamercolorbox}}%
  \vskip0pt%
}
\makeatother


\title{Team 40: Mortality Prediction in ICU \ \newline CSE 6250 Final Project}
\author{Vincent La\footnote{vincent.la@gatech.edu}, Avi Ananthakrishnan\footnote{avinash.ananthakrishnan@gatech.edu}}
\institute{Georgia Tech}
\date{\today}

\begin{document}

\maketitle

\begin{frame}
\label{Presentation Outline 1}
\frametitle{Presentation Outline}
\begin{enumerate}
\item[1.] \textbf{Research Question}
\newline
\item[2.] Brief Review of Existing Literature and Background
\newline
\item[3.] Data
\newline
\item[4.] Theoretical Framework
\newline
\item[5.] Empirical Design
\newline
\item[6.] Results
\newline
\item[7.] Future Considerations
\end{enumerate}
\end{frame}

\begin{frame}
\label{Research Question}
\frametitle{Research Question}
\begin{itemize}
	\item Overall: Capitalization of School Quality onto Housing Prices
	\newline
	\item Specific: What is the causal effect of increase in school quality (as measured by standardized test scores) on local housing prices?
\end{itemize}
\end{frame}

\begin{frame} 
\frametitle{Research Question}
\begin{itemize}
	\item Why do we care?
	\newline
		\begin{itemize}
%. On average, a one standard deviation improvement in teacher VA in a single grade raises earnings by 1.3% at age 28.			
			\item We believe better schools/teachers improve students' future outcomes (Chetty et. al, 2013)
			\item What about homeowners?
			\newline
		\end{itemize}
	\item What is the true effect of schools on housing prices?
	\newline
		\begin{itemize}
			\item Be careful! Omitted Variables Bias problem!
			%Over Schools Districts and over space -- we're mainly concerned with over space, e.g. neighborhood characteristics
			\item Abundant existing literature that presents solutions to these problems
		\end{itemize}
\end{itemize}
\end{frame}

\begin{frame}
\label{Presentation Outline 2}
\frametitle{Presentation Outline}
\begin{enumerate}
\item[1.] Research Question
\newline
\item[2.] \textbf{Brief Review of Existing Literature and Background}
\newline
\item[3.] Data
\newline
\item[4.] Theoretical Framework
\newline
\item[5.] Empirical Design
\newline
\item[6.] Results
\newline
\item[7.] Future Considerations
\end{enumerate}
\end{frame}

\begin{frame}
\label{Literature Review}
\frametitle{Review of Existing Literature}
Instrumental Variables
\begin{itemize}
%1 standard deviation increase in school quality is associated with a 1.87\% increase in housing prices
\item Rosenthal (2003)
\newline
\end{itemize}

Regression Discontinuity
\begin{itemize}
\item Black (1999) popularized the RD approach

\item Kane, Reigg, and Staiger (2006) 

%Paris, France one standard deviation increase in school quality is associated with a 2\% rise in sales prices.
\item Fack and Grenet (2010)

\item Black and Machin (2010) -- Survey Study 
\end{itemize}
\end{frame}

\begin{frame}
\label{Literature Review}
\frametitle{Review of Existing Literature}
Instrumental Variables
\begin{itemize}
%1 standard deviation increase in school quality is associated with a 1.87\% increase in housing prices
\item Rosenthal (2003)
\newline
\end{itemize}

Regression Discontinuity
\begin{itemize}
\item Black (1999) popularized the RD approach

\item Kane, Reigg, and Staiger (2006) 

%Paris, France one standard deviation increase in school quality is associated with a 2\% rise in sales prices.
\item Fack and Grenet (2010)

\item Black and Machin (2010) -- Survey Study 
\end{itemize}
\end{frame}

\begin{frame}
\label{Background: BPS}
\frametitle{Background: Boston Public School District (BPS)}
Quick Facts:
\begin{itemize}
\item Population of Boston (2012): 636,479
\item Land Area (2010): 48.28 mi$^2$
%Does not include Charter Schools
\item Total Schools (2012 - 2013): 127
\item Total Students Enrolled in BPS: $\approx$ 57,000 (74 \% of student age population)
\item Total Schools Serving Fourth Grade: 71
\item Total Fourth Grade Students Enrolled in BPS: $\approx$ 4,000
\end{itemize}
\end{frame}

\begin{frame}
\label{Background: BPS}
\frametitle{Background: Boston Public School District (BPS)}
Registration Mechanism: System of Priorities
\begin{itemize}
\item[1] Sibling + Walk Zone
\item[2] Sibling
\item[3] Walk Zone \newline
\end{itemize}
Walk Zone (Elementary Schools): ``1 mile radius around the school, calculated by drawing a circle with 1-mile radius centered at the school and adding the entirety of any geocode the circle touches" (Pathak and Shi, 2013). 
\newline \newline
Any student could apply to any school within his/her walk zone, regardless of actual school zone. 
\end{frame}

\begin{frame}
\label{Background: BPS}
\frametitle{Background: Boston Public School District (BPS)}
Walk Zone Facts
\begin{itemize}
\item 50\% of each school's seats are reserved for walk zone students.
\item (2012-2013) 86\% of BPS families listed a walk zone school as one of their top three choices.
\newline
\end{itemize}

Charter Schools
\begin{itemize}
\item Not part of BPS, do not follow the same registration/priority mechanism (Lottery Based)
\item We exclude charter schools from this study. \newline
\end{itemize}

%Pilot schools remain under local jurisdiction and maintain the same basic tenure and salary guidelines as traditional schools, but have more flexibility to make operational and curricular decisions.
Pilot Schools
\begin{itemize}
\item Technically part of BPS, occupy the middle ground between traditional public schools and charter schools
\item Follows the same student assignment policy
\end{itemize}
\end{frame}

\begin{frame}
\label{Background: MCAS}
\frametitle{MCAS}
Massachusetts Comprehensive Assessment System (MCAS): 4th Grade Tests
\begin{itemize}
\item[1] Mathematics
\item[2] English Language Arts and Reading Comprehension (ELA) \newline
\end{itemize}

Performance Levels (Raw Score Totals)
\begin{itemize}
\item[1] Advanced (260 - 280)
\item[2] Proficient (240 - 259)
\item[3] Needs Improvement (220 - 239)
\item[4] Warning/Failing (200 - 219) 
\end{itemize}
\end{frame}

\begin{frame}
\label{Presentation Outline 3}
\frametitle{Presentation Outline}
\begin{enumerate}
\item[1.] Research Question
\newline
\item[2.] Brief Review of Existing Literature and Background
\newline
\item[3.] \textbf{Data}
\newline
\item[4.] Theoretical Framework
\newline
\item[5.] Empirical Design
\newline
\item[6.] Results
\newline
\item[7.] Future Considerations
\end{enumerate}
\end{frame}

\begin{frame}
\label{Data}
\frametitle{Data}
Housing Data
\begin{itemize}
\item 2009 - 2013 Boston Residential Sales provided by the City of Boston Assessing Department
\item Includes all residential real estate ``arms-length" transactions. Total: 20,932
\item Includes a large vector of housing characteristics \newline
\end{itemize}

School Data
\begin{itemize}
\item Massachusetts Department of Education Statewide Reports
\item MCAS Results from 1998 - 2013
\item Also obtained vectors of student/teacher characteristics
\end{itemize}
\end{frame}

\begin{frame}
\label{Data: Table of Means}
\frametitle{Data}
\begin{table}[H]
\tiny
\newcolumntype{C}{>{\centering\arraybackslash}X}% centering
\centering
\begin{tabularx}{\textwidth}{l C C C}\hline
 & (1) & (2) & (3) \\\
VARIABLES & Mean & SD & \# Obs. \\ \hline
 &  &  &  \\
School Characteristics \\\
&&&\\\
\hspace{5mm} ELA Proficient or Higher & 25.95 & 12.32 & 71 \\\
\hspace{5mm} ELA Advanced & 3.023 & 2.882 & 71 \\\
\hspace{5mm} ELA Proficient & 22.93 & 10.01 & 71 \\\
\hspace{5mm} ELA Needs Improvement & 47.35 & 7.090 & 71 \\\
\hspace{5mm} ELA Warning/Failing & 26.75 & 11.93 & 71 \\\
\hspace{5mm} ELA Composite Performance Index & 65.41 & 10.42 & 71 \\\
\hspace{5mm} ELA Student Growth Percentile & 44.16 & 11.96 & 68 \\\
\hspace{5mm} Math Proficient or Higher & 23.67 & 12.58 & 71 \\\
\hspace{5mm} Math Advanced & 5.945 & 4.997 & 71 \\\
\hspace{5mm} Math Proficient & 17.74 & 8.161 & 71 \\\
\hspace{5mm} Math Needs Improvement & 45.73 & 7.492 & 71 \\\
\hspace{5mm} Math Warning/Failing & 30.66 & 11.75 & 71 \\\
\hspace{5mm} Math Composite Performance Index & 65.00 & 10.45 & 71 \\\
\hspace{5mm} Math Student Growth Percentile & 49.94 & 13.21 & 68 \\\
&&&\\\
\hspace{5mm} Total Number of Classes & 111.3 & 74.17 & 71 \\\
\hspace{5mm} Average Class Size & 17.93 & 2.370 & 71 \\\
\hspace{5mm} Total Number of Students & 420.2 & 192.7 & 71 \\\
\hspace{5mm} Percentage Black & 31.96 & 20.19 & 71 \\\
\hspace{5mm} Percentage White & 14.57 & 14.87 & 71 \\\
\hspace{5mm} Percentage Male & 52.37 & 3.743 & 71 \\\
\hspace{5mm} Percentage Students Limited English & 32.42 & 17.67 & 71 \\\
\hspace{5mm} Percentage Students Low Income & 73.07 & 12.74 & 71 \\\
\hspace{5mm} Total Number of Teachers & 30.37 & 15.06 & 71 \\\
\hspace{5mm} Percentage of Teachers Licensed & 95.74 & 2.904 & 71 \\ \hline
\end{tabularx}
\end{table}
\end{frame}

\begin{frame}
\label{Data}
\frametitle{Data}
School Data
\begin{itemize}
\item Prior to 2001, 4th Grade MCAS also include SCI section. Excluded these results from our study
\newline
\item Primary measure of school quality is percentage of students scoring proficient or advanced
\newline
\item Combine Math and ELA performance by simply summing the two, consistent with Black (1999)
\end{itemize}
\end{frame}

\begin{frame}
\label{Presentation Outline 4}
\frametitle{Presentation Outline}
\begin{enumerate}
\item[1.] Research Question
\newline
\item[2.] Brief Review of Existing Literature and Background
\newline
\item[3.] Data
\newline
\item[4.] \textbf{Theoretical Framework}
\newline
\item[5.] Empirical Design
\newline
\item[6.] Results
\newline
\item[7.] Future Considerations
\end{enumerate}
\end{frame}

\begin{frame}
\label{Theoretical Background}
\frametitle{Theoretical Background}
Initial Hypothesis
\begin{itemize}
\item Increasing the quality of the best walk zone school is associated with higher housing prices \newline
\item Increasing the quality of multiple walk one schools (increase school choice) is associated with lower housing prices.
	\begin{itemize}
		\item This is perhaps unintuitive
		\item Explained by application of Tiebout Model
	\end{itemize}
\end{itemize}
\end{frame}

\begin{frame}
\label{Theoretical Background: Tiebout Model}
\frametitle{Theoretical Background: Tiebout Model}
\end{frame}

\begin{frame}
\label{Theoretical Background: Tiebout Model}
\frametitle{Theoretical Background: Tiebout Model}
\end{frame}

\begin{frame}
\label{Theoretical Background: Tiebout Model}
\frametitle{Theoretical Background: Tiebout Model}
\end{frame}


\begin{frame}
\label{Presentation Outline 5}
\frametitle{Presentation Outline}
\begin{enumerate}
\item[1.] Research Question
\newline
\item[2.] Brief Review of Existing Literature and Background
\newline
\item[3.] Data
\newline
\item[4.] Theoretical Framework
\newline
\item[5.] \textbf{Empirical Design}
\newline
\item[6.] Results
\newline
\item[7.] Future Considerations
\end{enumerate}
\end{frame}

\begin{frame}
\label{Empirical Design}
\frametitle{Empirical Regression Equation}
\begin{equation}
\label{PrimaryEquation}
lnPrice_{it} = \beta_0  + \beta_1 School \ Quality + \beta_2 \mathbf{X}_{it} +  \phi + \gamma + \mu_{it},
\end{equation}
\begin{itemize}
\item $ln Price$: Natural log of the sales price of the property
\item $School \ Quality$: Measure of local school quality
\item $\mathbf{X}_{it}$: Vector of property-related characteristics
\item $\phi$: Time Dummies
\item $\gamma$: Vector of Neighborhood and School Characteristics \newline
\end{itemize}

\pause
However, omitted variables/endogeneity Problems remain.
\end{frame}

\begin{frame}
\label{Empirical Design RD Diagram}
\frametitle{Empirical Design: Regression Discontinuity Approach}
\end{frame}

\begin{frame}
\label{Empirical Design}
\end{frame}

\begin{frame}
\label{Empirical Design}
\frametitle{Empirical Design: Regression Discontinuity Approach}
\end{frame}

\begin{frame}
\label{Empirical Design}
\frametitle{Empirical Regression Equation}
\begin{equation}
\label{BasicActualEquation}
lnPrice_{it} = \beta_0 + \beta_1 Test_i + \beta_2 X_{it} + K + \phi + \mu_{it},
\end{equation}
\begin{itemize}
\item $ln Price$: Natural log of the sales price of the property
\item $Test_i$: MCAS Result of a walk zone school for house $i$
\item $\mathbf{X}_{it}$: Vector of property-related characteristics
\item $\phi$: Year Dummies
\item $K$ Square Block Dummies \newline
\end{itemize}
\end{frame}

\begin{frame}
\label{Presentation Outline 6}
\frametitle{Presentation Outline}
\begin{enumerate}
\item[1.] Research Question
\newline
\item[2.] Brief Review of Existing Literature and Background
\newline
\item[3.] Data
\newline
\item[4.] Theoretical Framework
\newline
\item[5.] Empirical Design
\newline
\item[6.] \textbf{Results}
\newline
\item[7.] Future Considerations
\end{enumerate}
\end{frame}

\begin{frame}
\label{Results: Baseline Regression}
\frametitle{Results: Baseline Regression}
\begin{table}[H]
\tiny
\newcolumntype{C}{>{\centering\arraybackslash}X}% centering
\caption{Effect of MCAS Scores on Housing Prices. Dependent Variable: Natural Log of Sale Price. SE Clustered by Walk Zone Groups}
\label{MCAS baseline results}
\centering
\begin{tabularx}{\textwidth}{l C C C C C}\hline
 & (1) & (2) & (3) & (4) & (5) \\\
VARIABLES & All Houses & All Houses & 0.5 Sq. Mile Blocks & 0.4 Sq. Mile Blocks & 0.3 Sq. Mile Blocks\\ \hline
 &  &  &  &  &   \\\
Best \% Prof or Adv & 0.121* & 0.0840* & 0.0380* & 0.0414*** & 0.0503**  \\\
 & (0.0618) & (0.0450) & (0.0194) & (0.0129) & (0.0202) \\\
Constant & 12.60*** & 12.21*** & 12.14*** & 12.09*** & 12.07*** \\\
 & (0.169) & (0.138) & (0.0861) & (0.0716) & (0.0799) \\\
 &  &  &  &  &    \\\
Observations & 20,716 & 20,320 & 20,320 & 20,320 & 20,320  \\\
 R-squared & 0.034 & 0.404 & 0.733 & 0.746 & 0.747 \\\
 House Controls$^a$ & NO & YES & YES & YES & YES \\\
 Year FE & NO & YES & YES & YES & YES\\\
 Square Block FE& NO & NO & YES & YES & YES \\ \hline
\multicolumn{6}{c}{ Robust standard errors in parentheses} \\
\multicolumn{6}{c}{ *** p$<$0.01, ** p$<$0.05, * p$<$0.1} \\
\end{tabularx}
\begin{minipage}[t]{1\columnwidth}
{\tiny $^a$ House Controls include: total \# bedrooms, total \# bathrooms, total \# bathrooms$^2$, size (in sq ft.), age of building, age of building$^2$. These are the same controls that Black (1999) uses.}
\end{minipage}\tabularnewline
\end{table}
\end{frame}

\begin{frame}
\frametitle{Results: Specification Test}
\begin{table}[H]
\tiny
\newcolumntype{C}{>{\centering\arraybackslash}X}% centering
\caption{Specification Tests. Effect of MCAS Scores interacted with number of bedrooms on Housing Prices. Dependent Variable: Natural Log of Sale Price.}
\label{MCAS Bedroom Sensitivity}
\centering
\begin{tabularx}{\textwidth}{l C C C C C}\hline
 & (1) & (2) & (3) & (4) & (5)  \\\
VARIABLES & All Houses & All Houses & 0.5 Sq. Mile Blocks & 0.4 Sq. Mile Blocks & 0.3 Sq. Mile Blocks \\ \hline
 &  &  &  &  &   \\
(Best \% Prof or Adv)$\times$ & 0.117** & 0.0976** & 0.0674*** & 0.0683*** & 0.0798*** \\
(Dummy $\geq 2$ Bedrooms) & (0.0552) & (0.0481) & (0.0218) & (0.0146) & (0.0222) \\\\
(Best \% Prof or Adv)$\times$ & 0.0392 & 0.0596 & -0.00406 & -0.00508 & 0.00859 \\
(Dummy $\leq 1$ Bedrooms) & (0.0484) & (0.0428) & (0.0196) & (0.0137) & (0.0202) \\\\
Constant & 12.68*** & 12.29*** & 12.27*** & 12.23*** & 12.19*** \\
 & (0.133) & (0.127) & (0.0735) & (0.0656) & (0.0702) \\
 &  &  &  &  &  \\
Observations & 20,716 & 20,320 & 20,320 & 20,320 & 20,320 \\
 R-squared & 0.064 & 0.410 & 0.753 & 0.767 & 0.767 \\
 House Controls$^a$ & NO & YES & YES & YES & YES\\
 Year FE & NO & YES & YES & YES & YES\\
 Square Block FE& NO & NO & YES & YES & YES \\ \hline
\multicolumn{6}{c}{ Robust standard errors in parentheses} \\
\multicolumn{6}{c}{ *** p$<$0.01, ** p$<$0.05, * p$<$0.1} \\
\end{tabularx}
\begin{minipage}[t]{1\columnwidth}
{\tiny $^a$ House Controls include: total \# bedrooms, total \# bathrooms, total \# bathrooms$^2$, size (in sq ft.), age of building, age of building$^2$. These are the same controls that Black (1999) uses.}
\end{minipage}\tabularnewline
\end{table}
\end{frame}

\begin{frame}
\frametitle{Results}
\begin{table}[H]
\tiny
\newcolumntype{C}{>{\centering\arraybackslash}X}% centering
\caption{Effect of MCAS Scores on Housing Prices. Dependent Variable: Natural Log of Sale Price. Restrict Sample to where difference between Best \% Prof or Adv in WZ and Second Best \% Prof or Adv in WZ falls in the 75th to 100th percentile. SE Clustered by Walk Zone Groups}
\label{MCAS Big Diff}
\centering
\begin{tabularx}{\textwidth}{l C C C C C}\hline
 & (1) & (2) & (3) & (4) & (5) \\\
VARIABLES & All Houses & All Houses & 0.5 Sq. Mile Blocks & 0.4 Sq. Mile Blocks & 0.3 Sq. Mile Blocks \\ \hline
 &  &  &  &  &  \\\
Best \% Prof or Adv & 0.0490 & 0.0235 & 0.0458* & 0.0805*** & 0.0837*** \\\
 & (0.129) & (0.0687) & (0.0249) & (0.0133) & (0.0201)  \\\
Constant & 13.14*** & 12.34*** & 12.22*** & 12.08*** & 12.10*** \\\
 & (0.465) & (0.235) & (0.105) & (0.0638) & (0.0874)  \\\
 &  &  &  &  &   \\\
Observations & 6,221 & 6,170 & 6,170 & 6,170 & 6,170 \\\
 R-squared & 0.003 & 0.584 & 0.782 & 0.783 & 0.787  \\\
 House Controls$^a$ & NO & YES & YES & YES & YES\\\
 Year FE & NO & YES & YES & YES & YES \\\
 Square Block FE& NO & NO & YES & YES & YES \\ \hline
\multicolumn{6}{c}{ Robust standard errors in parentheses} \\
\multicolumn{6}{c}{ *** p$<$0.01, ** p$<$0.05, * p$<$0.1} \\
\end{tabularx}
\begin{minipage}[t]{1\columnwidth}
{\tiny $^a$ House Controls include: total \# bedrooms, total \# bathrooms, total \# bathrooms$^2$, size (in sq ft.), age of building, age of building$^2$. These are the same controls that Black (1999) uses.}
\end{minipage}\tabularnewline
\end{table}
\end{frame}

\begin{frame}
\frametitle{Results}
\begin{table}[H]
\tiny
\newcolumntype{C}{>{\centering\arraybackslash}X}% centering
\caption{Effect of MCAS Scores on Housing Prices. Dependent Variable: Natural Log of Sale Price. SE Clustered by Walk Zone Groups}
\label{MCAS WZ Control}
\centering
\begin{tabularx}{\textwidth}{l C C C C C}\hline
 & (1) & (2) & (3) & (4) & (5) \\\
VARIABLES & All Houses & All Houses & 0.5 Sq. Mile Blocks & 0.4 Sq. Mile Blocks & 0.3 Sq. Mile Blocks \\ \hline
 &  &  &  &  &   \\\
Best \% Prof or Adv & 0.128** & 0.0918** & 0.0452** & 0.0518*** & 0.0525** \\\
 & (0.0632) & (0.0464) & (0.0200) & (0.0151) & (0.0237)  \\\
Total \# WZ & -0.0392** & -0.0374*** & -0.0124 & -0.0203** & -0.00403  \\\
 & (0.0175) & (0.0141) & (0.0102) & (0.00953) & (0.0120)  \\\
Constant & 12.73*** & 12.32*** & 12.16*** & 12.14*** & 12.07*** \\\
 & (0.172) & (0.145) & (0.0878) & (0.0665) & (0.0703) \\\
 &  &  &  &  &    \\\
Observations & 20,716 & 20,320 & 20,320 & 20,320 & 20,320  \\\
 R-squared & 0.050 & 0.418 & 0.733 & 0.747 & 0.747 \\\
  House Controls$^a$ & NO & YES & YES & YES & YES \\\
 Year FE & NO & YES & YES & YES & YES \\\
 Square Block FE& NO & NO & YES & YES & YES  \\ \hline
\multicolumn{6}{c}{ Robust standard errors in parentheses} \\
\multicolumn{6}{c}{ *** p$<$0.01, ** p$<$0.05, * p$<$0.1} \\
\end{tabularx}
\begin{minipage}[t]{1\columnwidth}
{\tiny $^a$ House Controls include: total \# bedrooms, total \# bathrooms, total \# bathrooms$^2$, size (in sq ft.), age of building, age of building$^2$. These are the same controls that Black (1999) uses.}
\end{minipage}\tabularnewline
\end{table}
\end{frame}

\begin{frame}
\frametitle{Results}
\begin{table}[H]
\tiny
\newcolumntype{C}{>{\centering\arraybackslash}X}% centering
\caption{Effect of MCAS Scores on Housing Prices. Dependent Variable: Natural Log of Sale Price. SE Clustered by Walk Zone Groups}
\label{MCAS Max Sec WZ}
\centering
\begin{tabularx}{\textwidth}{l C C C C C}\hline
 & (1) & (2) & (3) & (4) & (5) \\\
VARIABLES & All Houses & All Houses & 0.5 Sq. Mile Blocks & 0.4 Sq. Mile Blocks & 0.3 Sq. Mile Blocks  \\ \hline
 &  &  &  &  &    \\\
Best \% Prof or Adv & 0.213*** & 0.169*** & 0.0615*** & 0.0652*** & 0.0667*** \\\
 & (0.0474) & (0.0380) & (0.0214) & (0.0156) & (0.0243) \\\
Sec \% Prof or Adv & -0.366*** & -0.305*** & -0.0804** & -0.0974*** & -0.0877***  \\\
 & (0.0434) & (0.0450) & (0.0331) & (0.0256) & (0.0215)  \\\
Total \# WZ & 0.0249* & 0.0128 & 0.00714 & 0.00209 & 0.0152 \\\
 & (0.0133) & (0.0128) & (0.0131) & (0.0103) & (0.0120)  \\\
Constant & 12.93*** & 12.44*** & 12.20*** & 12.21*** & 12.13***  \\\
 & (0.149) & (0.125) & (0.0923) & (0.0746) & (0.0745)  \\\
 &  &  &  &     \\\
Observations & 20,010 & 19,630 & 19,630 & 19,630 & 19,630 \\\
 R-squared & 0.209 & 0.519 & 0.736 & 0.750 & 0.750 \\\
  House Controls$^a$ & NO & YES & YES & YES & YES \\\
 Year FE & NO & YES & YES & YES & YES \\\
 Square Block FE& NO & NO & YES & YES & YES  \\ \hline
\multicolumn{6}{c}{ Robust standard errors in parentheses} \\
\multicolumn{6}{c}{ *** p$<$0.01, ** p$<$0.05, * p$<$0.1} \\
\end{tabularx}
\begin{minipage}[t]{1\columnwidth}
{\tiny $^a$ House Controls include: total \# bedrooms, total \# bathrooms, total \# bathrooms$^2$, size (in sq ft.), age of building, age of building$^2$. These are the same controls that Black (1999) uses.}
\end{minipage}\tabularnewline
\end{table}
\end{frame}

\begin{frame}
\frametitle{Results}
\begin{table}[H]
\tiny
\newcolumntype{C}{>{\centering\arraybackslash}X}% centering
\caption{Effect of MCAS Scores on Housing Prices. Dependent Variable: Natural Log of Sale Price.}
\label{MCAS Dummies}
\centering
\begin{tabularx}{\textwidth}{l C C C C C}\hline
 & (1) & (2) & (3) & (4) & (5)  \\\
VARIABLES & All Houses & All Houses & 0.5 Sq. Mile Blocks & 0.4 Sq. Mile Blocks & 0.3 Sq. Mile Blocks \\ \hline
 &  &  &  &  &   \\
 \underline{No. of Top Schools}&  &  &  &   &  \\\
$\geq 1$ & 0.204* & 0.112 & 0.0614 & 0.0901** & 0.0713 \\\
 & (0.122) & (0.0957) & (0.0479) & (0.0382) & (0.0596) ) \\\
$\geq 2$ & -0.323*** & -0.235*** & -0.0234 & -0.0773*** & -0.0379  \\\
 & (0.104) & (0.0848) & (0.0292) & (0.0273) & (0.0390)  \\\
Constant & 12.91*** & 12.46*** & 12.21*** & 12.17*** & 12.18*** \\\
 & (0.0812) & (0.0752) & (0.0694) & (0.0612) & (0.0672) \\\
 &  &  &  &  & \\\
Observations & 20,932 & 20,536 & 20,536 & 20,536 & 20,536 \\\
 R-squared & 0.067 & 0.420 & 0.734 & 0.748 & 0.748  \\\
 House Controls$^a$ & NO & YES & YES & YES & YES\\\
 Year FE & NO & YES & YES & YES & YES\\\
 Square Block FE& NO & NO & YES & YES & YES \\ \hline
\multicolumn{6}{c}{ Robust standard errors in parentheses} \\
\multicolumn{6}{c}{ *** p$<$0.01, ** p$<$0.05, * p$<$0.1} \\
\end{tabularx}
\begin{minipage}[t]{1\columnwidth}
{\tiny $^a$ House Controls include: total \# bedrooms, total \# bathrooms, total \# bathrooms$^2$, size (in sq ft.), age of building, age of building$^2$. These are the same controls that Black (1999) uses.}
\end{minipage}\tabularnewline
\end{table}
\end{frame}

\begin{frame}
\label{Presentation Outline 7}
\frametitle{Presentation Outline}
\begin{enumerate}
\item[1.] Research Question
\newline
\item[2.] Brief Review of Existing Literature and Background
\newline
\item[3.] Data
\newline
\item[4.] Theoretical Framework
\newline
\item[5.] Empirical Design
\newline
\item[6.] Results
\newline
\item[7.] \textbf{Future Considerations}
\end{enumerate}
\end{frame}

\begin{frame}
\label{Conclusion}
\frametitle{Conclusion}
\begin{itemize}
\item[1] One SD increase in best walk zone school's \% students scoring proficient or advanced on MCAS $\Rightarrow$ 5\% increase in house sales price
	\begin{itemize}
		\item Average Sales Price: \$535,645 $\approx$ \$26,000 premium \newline
	\end{itemize}
\item[2] Presence of several good schools within walk zone mitigates (and reverses) this effect 
	\begin{itemize}
		\item Holding best school constant, improving second best school within walk zone leads to a decrease in sales price
		\item Home buyers willing to pay 6 - 9\% premium for houses with 1 good walk zone school. Only 2 - 4 \% for houses with multiple good walk zone schools.
	\end{itemize}
\end{itemize}
\end{frame}


\begin{frame}
\fontsize{2}{1}
\frametitle{References}
 \begin{enumerate}
	\item Angrist, J. D., Cohodes, S. C., Dynarski, S. M., Pathak, P. A., and Walters, C.R. (2013). Stand and Deliver: Effects of Boston's Charter High Schools on College Preparation, Entry, and Choice. {\em NBER Working Paper,} No. 19275. 
	
	\item Abdulkadiroglu, Atila; Pathak, Parag A.; Roth, Alvin E. and Siinmez, Tayfun. (2005) "The Boston Public School Match." \textit{American Economic Review, 2005 (Papers and Proceedings)}, 95(2), pp. 368-71. 

	
	\item Abdulkadiroglu, A., Angrist, J., Dynarski, S., Kane, T.J., and Pathak, P. (2011). Accountability and flexibility in public schools: Evidence from Boston's charters and pilots. \textit{The Quarterly Journal of Economics}, 126(2):699-748
	
	\item Belfield, C. R., and Levin H. M. (2002). The Effects of Competition Between Schools on Educational Outcomes: A Review for the United States. \textit{Review of Educational Research} 72(2): 279-341

	\item Black, Sandra E. (1999). ``Do Better Schools Matter? Parental Valuation of Elementary Education." \textit{Quarterly Journal of Economics} 114(2): 577-599
	
	\item Black, Sandra E. and Machin, Stephen (2010). ``Housing Valuation of School Performance." \textit{Handbook of the Economics of Education} Vol 3. Chapter 10: 485-519
	
	\item Budde, R. (1988) Education by Charter: Restructuring school districts: Key to long-term continuing improvement in American Education. Andover, MA: Regional Laboratory for Educational Improvement of the Northeast \& Islands
	
	\item Crone, Theodore. (2006). Capitalization of the Quality of Local Public Schools: What do Home Buyers Value? \textit{Federal Reserve Bank of Philadelphia Working Paper} No. 06-15.
	
	\item Gibbons, S., Machin, S., (2006). Paying for primary schools: admission constraints, school popularity or congestion? \textit{Economic Journal} 116(510), 77-92
	
	\item Hanushek, E. A., Kain, J. F., Rivkin, S. G., and Branch, G. F. (2007) Charter School Quality and Parental Decision Making with School Choice. \textit{Journal of Public Economics} 91(5): 823-848
	
	\item Horowitz, J., Keil, S., and Spector, L. (2009) Do Charter Schools Affect Property Values?. \textit{The Review of Regional Studies} 39(3): 297-316
	
	\item Imberman, S., Naretta, M., and O'Rourke, M. (2014). The Value of Charter Schools: Evidence from Housing Prices.
	
	\item Nguyen-Hoang, P., and Yinger, J. (2011). The Capitalization of School Quality into House Values: A Review. \textit{Journal of Housing Economics} 20, pp.30-48
	
	\item Pathak, P., and Shi, P. (2013). Simulating Alternative School Choice Options in Boston -- Main Report. \textit{MIT School Effectiveness and Inequality Initiative}
	
	\item Rouse, C. E., and Barrow, L. (2009). School Vouchers and Student Achievement: Recent Evidence and Remaining Questions. \textit{Annual Review of Economics} 1: 17-24
	
	\item Rosenthal, L. (2003), The Value of Secondary School Quality. \textit{Oxford Bulletin of Economics and Statistics} 65: 329–355
	
	\item Ross, S., Yinger, J. (1999). Sorting and Voting: A Review of the Literature on Urban Public Fiannce. \textit{Handbook of Urban and Regional Economics} 3, pp. 2001-2060
	
	\item Schwartz, A., Voicu, I., and Mertens K. M., (2014). Do Choice Schools Break the Link Between Public Schools and Property Values? Evidence from House Prices in New York City.
	
	\item Thernstrom, A. M. and Thernstrom, S. (2003). No excuses: Closing the racial gap in learning. Simon \& Schuster, New York.
	
	\item Weimer, D. L., Wolkoff, M. J., (2001). School performance and housing values: using non-contiguous district and incorporation boundaries to identify school effects. \textit{National Tax Journal} 54(2): 231 - 254
  
  \end{enumerate}
\end{frame}

\begin{frame} 
\frametitle{Questions?}
\centerline{{\Huge Questions?}}
\end{frame}

\end{document}

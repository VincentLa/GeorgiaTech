\documentclass[12pt, final]{article}
\usepackage{fullpage,amsthm,amsfonts,amsmath}
\usepackage{enumerate}
\usepackage[margin=1in]{geometry}
\usepackage{color}
\usepackage{epsfig}
\usepackage{epstopdf}
\usepackage{datatool}
\usepackage{array}
\usepackage{tabu}
\usepackage{amsmath}
\usepackage{caption}
\usepackage{float}
\usepackage{tabularx, booktabs}
\usepackage{pdfpages}
\usepackage{standalone}
\usepackage{hyperref}
\usepackage{subfigure}
\usepackage{graphicx}
\hypersetup{colorlinks = true, urlcolor = blue, filecolor = blue, linkcolor = blue}
\floatstyle{plaintop}
\restylefloat{table}
\renewcommand{\thefootnote}{$\star$} 

\usepackage{csquotes}
\usepackage[style=apa]{biblatex}
% \DeclareLanguageMapping{english}{english-apa}% no longer needed if you have recent (La)TeX installation
\addbibresource{biblatex-examples.bib}% \bibliography{...} is legacy format

\begin{document}
\title{Collecting Your Sources (Spring 2019) Assignment 5}

\date{\today}

%%
\renewcommand{\thefootnote}{$\dag$}
%%
\author{Vincent La\footnote{vincent.la@gatech.edu}}

\maketitle

\begin{abstract}
Qualifier Question we will answer: You have narrowed your topic to an application that will allow K-12 students to be pen pals with each other more seamlessly across diverse areas of the country or even the globe that will filter out messages that are egregious, harmful, or hateful, but let's back up for a minute.  What societal benefits are gained from social learning through social media? What are the most successful ways of improving knowledge through social learning in the social media context? What are the risks of not filtering egregious, harmful, or hateful messages? What are the risks or effects of filtering these messages? Is there any impact on student engagement? Remember, the goal of the qualifier question is to prove you're ready to contribute to this community. So, you'll want your answer to be a bit longer and deeper, and especially well-referenced.
\end{abstract} 

\newpage
%%%%%
\renewcommand{\thefootnote}{\number\value{footnote}} 
%%%%%

\section{Qualifier Question Response}

The main purpose of this application is to encourage not only young children (K-12), but also the general population to participate in social learning through an online application or ``social media". The application will connect people with others in different communities to facilitate learning through engaging in conversations with other people from different backgrounds who offer different perspectives. To discuss the societal benefits that are gained from social learning through social media, I'll start by answering the converse; that is, how is the status quo currently detrimental to society? Currently, our inequity in our society is growing to unprecedented levels, Piketty (2015). What this means is that children are more and more growing up in communities where they do not have exposure to what life is like in other communities. As children grow up into adults, their attitudes become even more solidified. This attitude breeds cynicism, discrimination, and xenophobia. For example, Colleoni (2014) examine this in depth from a political perspective on Twitter. What the authors find is that people, regardless of whether they identify as Democratic or Republican, often agree with a politician who identifies as such, even if their views are not necessarily Democratic or Republican. What this facilitates is a phenomenon often referred to as an ``echo chamber". That is, people only see the views and perspectives of what they expect and they keep only hearing those views. They come to believe only what they want and any other views are likely to be shunned, which further drives xenophobia. While Colleoni (2014) discuss this in the world of politics on mass social media, like Twitter, similar phenomenon is happening even with children. Ditto and Koleva further describe the differences between American liberals and conservatives in terms of a lack of moral empathy. Often, the difficulty to relate to others’ points of view comes from a lack of understanding of the visceral, conditioned responses that are outside of an individual’s control.  For example, liberals often cannot relate to extreme conservatives' visceral distaste for ``nontraditional" sexuality, while conservatives often cannot relate to liberals’ strong desires to address prejudice at the expense of personal freedoms. While this moral empathy gap is hard to bridge, Ditto and Koleva argue that recognizing and appreciating these affective differences is a good first step to alleviating the culture war (Ditto, 2011). 
\\
\\
There is also enumerous literature that speaks to general successful ways to improve knowledge through social learning in a social media context. For example, Dabbagh and Kitsantas (2012), speak to Personal Learning Environments as a promising pedagogical approach for integrating both formal and informal learning using social media. Learning on demand is becoming a type of lifestyle in modern society (McLoughlin and Lee, 2007). Learners of any age are now constantly seeking information whether it be at work, school, or just satisfying and curiosity. Furthermore, we live in an unprecedented age where all information lies at our fingertips. Thus, learners are not just passive consumers of information, we are all co-creators in it. Learning in the context of social media has become self-motivated, active, and informal, as people learn and teach through contributions to conversation threads and other forums. In Dabbagh and Kitsantas (2012), they find that there is strong evidence that social media can facilitate the creation of PLEs that help learners aggregate and share the results of learning achievements, participate in collective knowledge generation, and manage their own meaning making. Furthermore, social media is not only affecting young children and students. In fact, the gap between usage in social media between young adults and the more elderly is shrinking (Caruso and Salaway, 2007). Successful social media platforms such as blogging platforms (e.g. Wordpress), Twitter, and wiki software all allow people to teach and learn actively via social learning and social media. 
\\
\\
Overall, these new social media platforms are creating new ``Personal Learning Environments" within social media. Martindale and Dowdy (2010) posit that Personal Learning Environments are an outcome of the tools that social media has provided learners enabling them to create, organize, and share content. Furthermore, they are more and more becoming increasingly effective in addressing issues of personalization that are often absent in traditional institutional learning environments. They find that traditional institutional learning environments promote tools to facilitate teaching as opposed to tools to promote student engagement, which is subtly different but important as the latter has been shown to be more effective at moving educational outcomes. In the physical world, learners typically rely on lunchtime discussions, student organizations, brown bag sessions and study groups for peer support and informal learning networks, but now online Personal Learning Environments are providing such sessions at scale. 
\\
\\
On the application of NLP, I think that filtering egregious, harmful, or hateful messages is quite important. Kou (2013) explore this topic in the context of a very popular video game, League of Legends. What they find is that because players are hidden behind the facade of a screen, they are more likely to display toxic behavior and lash out at other users. Because it might seem so impersonal, players are connected through the internet, there seems to be less filters. However, players on the receiving end have an extremely negative experience. Thus, Riot Games has taken upon itself to make sure to filter out those messages to ensure better behavior. Further risks are that if we don't filter hateful messages, our application could have the negative effect of not protecting at-risk communities, young women, etc. from being harassed. Of course, there is some risk of filtering messages, especially if we're filtering messages that aren't actually hateful. That could result in censoring, but it's definitely better to have false positives than false negatives. 

  \begin{thebibliography}{1}
  \bibitem{Caruso} Caruso, J. B., \& Salaway, G. (2007). The ECAR study of undergraduate students and information technology, 2007. Retrieved December, 8, 2007.
  \bibitem{Ceprano} Ceprano, M. A., \& Garan, E. M. (1998). Emerging voices in a university pen‐pal project: Layers of discovery in action research. Literacy Research and Instruction, 38(1), 31-56.
  \bibitem{Colleoni} Colleoni, E., Rozza, A., \& Arvidsson, A. (2014). Echo chamber or public sphere? Predicting political orientation and measuring political homophily in Twitter using big data. Journal of Communication, 64(2), 317-332.
  \bibitem{Dabbagh} Dabbagh, N., \& Kitsantas, A. (2012). Personal Learning Environments, social media, and self-regulated learning: A natural formula for connecting formal and informal learning. The Internet and higher education, 15(1), 3-8.
  \bibitem{Ditto} Ditto, P. H., \& Koleva, S. P. (2011). Moral empathy gaps and the American culture war. Emotion Review, 3(3), 331-332.
 \bibitem{Kou} Kou, Y., \& Nardi, B. (2013). Regulating anti-social behavior on the Internet: The example of League of Legends.
  \bibitem{Lie} Lie, W. W., \& Yunus, M. M. (2018). Pen Pals Are Now in Your Finger Tips—A Global Collaboration Online Project to Develop Writing Skills.
  \bibitem{Martindale} Martindale, T., \& Dowdy, M. (2010). Personal learning environments. Emerging technologies in distance education, 177-193.
  \bibitem{McLoughlin} McLoughlin, C., \& Lee, M. J. (2007, June). Listen and learn: A systematic review of the evidence that podcasting supports learning in higher education. In EdMedia+ Innovate Learning (pp. 1669-1677). Association for the Advancement of Computing in Education (AACE).
  \bibitem{Piketty} Piketty, T. (2015). About capital in the twenty-first century. American Economic Review, 105(5), 48-53.
  \end{thebibliography}


\end{document}
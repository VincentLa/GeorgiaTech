\documentclass[12pt, final]{article}
\usepackage{fullpage,amsthm,amsfonts,amsmath}
\usepackage{enumerate}
\usepackage[margin=1in]{geometry}
\usepackage{color}
\usepackage{epsfig}
\usepackage{epstopdf}
\usepackage{datatool}
\usepackage{array}
\usepackage{tabu}
\usepackage{amsmath}
\usepackage{caption}
\usepackage{float}
\usepackage{tabularx, booktabs}
\usepackage{pdfpages}
\usepackage{standalone}
\usepackage{hyperref}
\usepackage{subfigure}
\usepackage{graphicx}
\hypersetup{colorlinks = true, urlcolor = blue, filecolor = blue, linkcolor = blue}
\floatstyle{plaintop}
\restylefloat{table}
\renewcommand{\thefootnote}{$\star$} 

\usepackage{csquotes}
\usepackage[style=apa]{biblatex}
% \DeclareLanguageMapping{english}{english-apa}% no longer needed if you have recent (La)TeX installation
\addbibresource{biblatex-examples.bib}% \bibliography{...} is legacy format

\begin{document}
\title{Collecting Your Sources (Spring 2019) Assignment 5}

\date{\today}

%%
\renewcommand{\thefootnote}{$\dag$}
%%
\author{Vincent La\footnote{vincent.la@gatech.edu}}

\maketitle

\begin{abstract}
This assignment provides an annotated bibliography on the most pertinent literature for the project area. Similar to previous assignments, there will be an application of Natural Langage Processing in Education. However, whereas before I was thinking of focusing specifically on detecting fake news, I've pivoted now to do more a Development track project. The aim is to build an application that will allow K-12 students to be pen pals with each other more seamlessly across diverse areas of the country or even the globe. The goal is to expose young students to different parts of the world and not be stuck in an "echo chamber". I think NLP may still be an application by using NLP to filter out messages that are egregious, harmful, or hateful. In summary, the main project I want to focus on now is an application that will allow K-12 students to be pen pals with other peers across the globe. The research I will focus on are: Effect of Pen Pals on Educational Outcomes, How to Teach Empathy, and Natural Language Processing in Educational Learning.
\end{abstract} 

\newpage
%%%%%
\renewcommand{\thefootnote}{\number\value{footnote}} 
%%%%%

\section{Annotated Bibliography}

\begin{enumerate}

\item \textbf{Baker, D. F. (2017). Teaching Empathy and Ethical Decision Making in Business Schools. Journal of Management Education, 41(4), 575-598.}

In this paper, the author talks about teaching empathy and ethical decision making in Business Schools. While the application that I want to build is targeted more at K-12 students, I found this article relevant since overall the goal of my application is to teach empathy for students which can also inform ethical decision making. So I wanted to research what this looked like in other populations, e.g. business school. The authors sought to understand how individuals respond to the ethical dilemmas in their lives. In any given situation, multiple social and psychological variables interact to influence ethical decision making. The purpose of this article is to explore how one such variable, empathy, affects the ethical decision-making process and to identify specific instructional strategies that both increase empathy and challenge students to consider the role that empathy plays in their own decisions.
\\
\\
To test this, the authors designed two different learning activities. In the first, students recommended median salaries for several positions in a fictitious company and then use those salaries to create two family budgets, one for an entry-level position and another for the CEO. The second activity measures students' empathic concern and encourages students to consider the relationship between empathic concern and decision making in business. Increased awareness of self and others prompts a more deliberate, thoughtful decision-making process when assessing ethical situations. This finding is very important to me because it shows that teaching empathy results in better decision making. While this study was done in the context of business school, I strongly believe that if we were to teach empathy in young elementary school students it would propel them into the future of being more empathetic, consider different view points and make better decision. 

\item \textbf{Burstein, J. (2009, March). Opportunities for natural language processing research in education. In International Conference on Intelligent Text Processing and Computational Linguistics (pp. 6-27). Springer, Berlin, Heidelberg.}

This research paper is about opportunities for natural language processing research in education. While the core of my project is around building an application for pen-pal, I think NLP will be important to screen messages and remove hateful language. As an example, the authors speak to how NLP is being used to automate the scoring of student texts with respect to linguistic dimensions such as grammatical correctness or organizational structure. As another example, dialogue technologies are being used to achieve the benefits of human one-on-one tutoring - particularly in STEM domains - in a cost-effective and scalable manner. Another example include processing text from the web in order to personalize instructional materials to the interests of individual students, automate the generation of test questions for teachers, or (semi-)automate the authoring of an educational technology system. The third point is similar to what I want to build to detect language and personalize materials. The second is interesting because potentially we could build an NLP AI such that there could be an AI that facilitates conversation. 

\item \textbf{Camp, C., \& Camp, L. (2018). Teaching Empathy and Conflict Resolution to People with Dementia: A Guide for Person-centered Practice. Jessica Kingsley Publishers.}

This paper is part of research done in terms of how to teach empathy especially in the context of conflict resolution. While this paper is not directly the same as what I would want to build as it actually focuses on dementia patients whereas the population I want to affect is K-12, I found it to be relevant since one of the main goals I want to pursue in building this app is to teach empathy and I found it interesting that their focus was on conflict resolution. In this book, the authors speak to how teaching empathy can result in lessons for conflict resolution. I would hope that my own application can teach students lessons broader than just writing skill and things like conflict resolution. 

\item \textbf{Ceprano, M. A., \& Garan, E. M. (1998). Emerging voices in a university pen‐pal project: Layers of discovery in action research. Literacy Research and Instruction, 38(1), 31-56.}

This paper is part of research done to look at the effect of a Pen Pal program on learning. The authors investigate a semester-long pen-pal project at 18 participating university language arts students, a class of first graders and their teacher, and the university instructors to become co‐learners in meaningful, action research. The weekly letters the university students exchanged with a class of first‐graders became the springboard for discussion. Insights about writing as derived by all participants are discussed and are accompanied with exemplary samples of children's emergent literacy. An assessment rubric created to chart children's growth in voice as well as the technical aspects of their writing is presented.
\\
\\
The authors find that children's rowth in voice as well as the technical aspects of their writing improved. In addition, excerpts from the students' journals show that they were more aware of the effects of mediation on their own writing voice. In addition, the impact of the project with university students showed that they came to value systemic data collection. I found this interesting because it shows that the benefits of Pen Pal programs is beyond just improving writing, it could affect other aspects of life. This is important for my application since it motivates why I would want to build it.

\item \textbf{Chen, A., Hanna, J. J., Manohar, A., \& Tobia, A. (2018). Teaching Empathy: the Implementation of a Video Game into a Psychiatry Clerkship Curriculum. Academic Psychiatry, 42(3), 362-365.}

This article is very interesting because it looks at a very novel way to teach empathy. The authors use a video game to teach empathy in a psychiatry clerkship curriculum. Again, this paper is focused on medical literature and medical students although I think there are applications to what I am trying to build. The literature demonstrates that it often deteriorates during medical school. The purpose of this study is to investigate the use of the interactive video game “That Dragon, Cancer” as a tool to teach empathy to third-year medical students. The article found that the average level of empathy in the students as measured by least square means of answers to the truncated JSPE improved after playing video games. I found this paper interesting due to the quantitative ways they actually measured empathy and I wonder if I can build this quantitative measurement into the application I want to build.

\item \textbf{Chen, Y., Zhou, Y., Zhu, S., \& Xu, H. (2012, September). Detecting offensive language in social media to protect adolescent online safety. In Privacy, Security, Risk and Trust (PASSAT), 2012 International Conference on and 2012 International Confernece on Social Computing (SocialCom) (pp. 71-80). IEEE.}

I chose this article because it is exactly what I would want to do for NLP to detect offensive language in my application to protect adolescent online safety. Given that the target audience of my application is K-12 students, this is very important to me and I want to make sure that we can flag offensive language before subjected end users to hateful speech. The authors propose the Lexical Syntactic Feature (LSF) architecture to detect offensive content and identify potential offensive users in social media. The authors distinguish the contribution of pejoratives/profanities and obscenities in determining offensive content, and introduce hand-authoring syntactic rules in identifying name-calling harassments. In particular, the authors incorporate a user's writing style, structure and specific cyber bullying content as features to predict the user's potentiality to send out offensive content. Results from experiments showed that our LSF framework performed significantly better than existing methods in offensive content detection. It achieves precision of 98.24\% and recall of 94.34\% in sentence offensive detection, as well as precision of 77.9\% and recall of 77.8\% in user offensive detection. This is extremely interesting and potent for the application I am trying to build.

\item \textbf{Colleoni, E., Rozza, A., \& Arvidsson, A. (2014). Echo chamber or public sphere? Predicting political orientation and measuring political homophily in Twitter using big data. Journal of Communication, 64(2), 317-332.}

This paper investigates political homophily (tendency for people to seek out or be attracted to people similar to themselves) on Twitter. I picked this paper out as another source to speak about the echo chamber. However, this finding is also relevant for my project since children K-12 are most likely to be affected by peer pressure and want to conform to their friend group and community. This paper shows the affects of the echo chamber. Using a combination of machine learning and social network analysis the authors classify users as Democrats or as Republicans based on the political content shared. The paper then investigate political homophily both in the network of reciprocated and nonreciprocated ties. In general, Democrats exhibit higher levels of political homophily. But Republicans who follow official Republican accounts exhibit higher levels of homophily than Democrats. In addition, levels of homophily are higher in the network of reciprocated followers than in the nonreciprocated network. Thus, this finding is important since it will be important to target children in areas especially that are in Republican areas to expose children to different view points and not get stuck in the echo chamber.

\item \textbf{Ding, Y., Cheung, B., Kong, T., Lee, W. Y., \& Yao, J. (2017). Grandpal Penpals: A qualitative study of a social program on senior quality of life in residential care facilities. Medical Journal UBCMJ, 8(2), 16-20.}

This is a paper that also looks at the effect of pen pal programs on outcomes. However, this paper is unique in that it isn't specifically looking at the K-12 population. Actually quite the opposite, this paper looks at the program targeted to seniors living in residential care but connect the seniors to a nearby elementary school. A common belief is that seniors have a lower quality of life in residential care facilities. The authors qualitatively explored how this program related to the senior participants’ quality of life in the domains of motivation, activities, relationships, and autonomy. Overall the conclusion was while participants greatly enjoyed GP, they perceived little relation of the program to their overall quality of life. Our research suggests that other highly engaging, goal–oriented, long–term social programs with increased senior–senior or senior–family interaction may be of greater relevance and benefit. Again this is pertinent to me because even though the results weren't conclusively positive, learning what doesn't work is just as important since it informs what the app should and should not be built for. Again, this signals to me that match-making and how we actually connect users is very important part of the application. 

\item \textbf{Everhart, R. S. (2016). Teaching tools to improve the development of empathy in service-learning students. Journal of Higher Education Outreach and Engagement, 20(2), 129-154.}

This paper is directly relevant to what I'm trying to do in the sense that the article talks about teaching tools used to improve development of empathy in students. Students participating in service-learning classes experience many benefits, including cognitive development, personal growth, and civic engagement. This article describes a project designed to pilot teaching tools (e.g., self-assessment, reflective writing) related to empathy development in 12 undergraduate students. This is important since the tools are actually somewhat similar to the application I want to write. Self-assessments and reflective writing are all things that could be features of the application as well, and I would be interesting in learning more about how users would engage with this. The study found that the level of student empathy increased across semesters. Findings suggest that observing the emotional experiences of others enhance this learning, which has major impact on what I want to do since my application is all about connecting peers across the country and globe together. 

\item \textbf{Ferreira, R., Kovanovic, V., Gasevic, D., Rolim, V. (2018, June). Towards Combined Network and Text Analytics of Student Discourse in Online Discussions. In International Conference on Artificial Intelligence in Education (pp. 111-126). Springer, Cham.}

The next paper I bring up is applications in Natural language Processing in Education. The reason why this is important is because I think NLP will be a critical part in the application. For example, while I would hope that intentions of users are good, I would want to build some NLP classifier so that if it detects toxic language or racist, harmful language I would want to remove it before it can be sent. In Ferreira (2018) the authors look at the usge of NLP and Topic Modeling to evaluate students' use of asynchronous discussion (e.g. on an online platform like Piazza) in online learning environments. They look at how students' cognitive development increase across different course topics. This is important since I'm sure we can use similar techniques in our NLP. 

\item \textbf{Friedland, E. S., \& Truesdell, K. S. (2004). Kids reading together: Ensuring the success of a buddy reading program. The Reading Teacher, 58(1), 76-79.}

This paper is similar to the topic of researching the effect of pen-pals. Instead of Pen-pals however, the intervention here was a "buddy reading" program. Buddy reading program is a program where students read together. The purpose of such a program was to help teachers foster positive attitudes towards reading. This is especially important with school districts cutting resources so we need to find more innovative ways to support reading success in children. The authors say that they believe that the buddy reading program helped and established success in supporting positive attitudes towards reading. This is pertinent to my project since again it shows that peer-learning, in this case through reading, helps improve educational outcomes.

\item \textbf{Hatfield, E., \& Rapson, R. L. (2015). From pen pals to chat rooms: the impact of social media on Middle Eastern Society. SpringerPlus, 4(1), 254.}

This paper belongs in the set of research that I'm doing around efficacy of Pen Pals. In this case, this is the impact of social media on Middle Eastern Society so this is pretty similar in the sense of it's looking at the impact of an online application. In this article, the authors discuss the use of social media to communicate in the Middle East. They find that prevalence is more common than they expected and discuss the reasons why men and women are using the internet and social media. They find that both men and women use social media to consume news and learn information as well as build relationships. In fact, the authors focus on the impact of such media on men's and women's relationships—including cross-gender friendships, romantic relationships, and sexual relationships. Thus, I would say that an application like the one I am planning on building to connect students across the world could have demand even in foreign countries. 

\item \textbf{Jamieson, K. H., \& Cappella, J. N. (2008). Echo chamber: Rush Limbaugh and the conservative media establishment. Oxford University Press.}

The concept of an echo chamber, especially in politics, has been a particularly relevant concept in today's society due to the rise of social media and inequality on the rise. Because of the way that media is propagated today, people are more likely to only listen to the news that they want to. In addition, due to rising inequality, people are likely to stay in their own communities and are not exposed to different view points. The issue with this is that people get stuck in an "echo chamber" where they are not exposed to other view points and thus view others as "opposition" and this means that people are not willing to empathize with others. In this book specifically, the author touches on these points, especially with respect to the implications of the emergence of mass audience, ideologically coherent, conservative opinion media and how it attacks the democratic opposition. Fox and Limbaugh insulate their audience from persuasion by Democrats by building up a body of opinion and evidence that makes Democratic views seem alien and unpalatable. I think this is particularly applicable to my project because political echo chamber is very relevant for kids as well. Children in schools are exposed to only the communities that they are in. Many children are not exposed to other environments. This is harmful, as shown by Jamieson (2008) and may have a profound effect on their future. Thus, this means that our application to build a way for children to communicate more easily is important.

\item \textbf{Kirshner, J., Tzib, E., Tzib, Z., \& Fry, S. (2016). From Pen Pals to Global Citizens. Educational Leadership, 74(4), 73-74.}

I picked out this paper since the application that I want to build is an electronic application to allow children to write to each other, similar to the concept of "Pen-pals". So this paper belongs to the set of research around efficacy of Pen Pals. The authors compile a set of articles that describe three projects aimed at offering students authentic opportunities to develop global competencies. The first article describes Out of Eden Learn, an initiative from Project Zero at Harvard Graduate School of Education. The project engages students in learning journeys that follow Pulitzer Prize-winning journalist and National Geographic Fellow Paul Salopek as he walks around the world. The second article highlights a Project-Based Inquiry Global unit that connects students in North Carolina with those in China as they jointly explore water ecology. The final article tells about a collaboration between classrooms in Colorado and Belize that serves as both a professional development opportunity for teachers and a cultural learning exchange for students. The authors find that the students really benefited in these projects to develop global competencies. I find this paper to be applicable because I am trying to do something similar by building an application to expose students to other parts of the country and not get stuck in their own ``echo chamber" and get students to be more empathetic to enhance their learning.

\item \textbf{Kou, Y., \& Nardi, B. (2013). Regulating anti-social behavior on the Internet: The example of League of Legends.}

I found this paper to be very interesting and unique since it's around the usage of Natural language processing to regular anti-social behavior on the internet, in the context of the video game, League of Legends. League of Legends is a video game that is played 5v5 and the client performs match making to pair a team of 5 against another team of 5. However, in the game, players can type whatever they want to facilitate team work, but it can be harmful when players type hateful messages to other players. Riot Games developed a machine learning algorithm to detect such toxic behavior, flag it, and ban users for using the language. I think this is important since it's very similar in concept what we would want to do in our application to build guardrails in the application to prevent such bad behavior. 

\item \textbf{Laughey, W., Sangvik Grandal, N., Stockbridge, C., \& Finn, G. M. (2018). Twelve tips for teaching empathy using simulated patients. Medical Teacher, 1-5.}

This is a paper around how to teach empathy using simulated patients. While this paper is more geared towards medical students I found this paper to be interesting especially since one of the key goals of my application is to teach empathy via connection of peers from diverse areas. The paper talks about how teaching medical students empathy is hard and not currently a focus of medical schools. The paper discusses tips on how to build empathy such as attentive listening. I think that I can learn a lot from this paper and apply these lessons to my own application, even though I'm not focused on medical students. For example, to focus on ``attentive listening" perhaps I would design my app so that when responding, it would prompt the peer to specifically engage in their peer's responses so that it facilitates the conversation and forces the children to ``listen" to each other and respond. 

\item \textbf{Lie, W. W., \& Yunus, M. M. (2018). Pen Pals Are Now in Your Finger Tips—A Global Collaboration Online Project to Develop Writing Skills.}

This is another paper that looks at the effect of Pen Pal program on outcomes. In this particular paper, the authors look at a foreign country, Malaysia and look at the effect of the Pen Pal program. This is important to me because I would hope that my application could have global impact. The authors note that many Malaysian students have an aversion towards writing. This study explores the potential of using an online material, in the English as Second Language (ESL) classroom to develop writing skill with peers around the world via an online collaborative project. Thirty 12-year-old primary school students from Cheras, Selangor were chosen as the participants. After the March test, they were required to join PenPal Schools and take part in the online collaborative project.  The research used mixed-method design where quantitative data from pre- and post-tests and responses from semi-structured interviews were used to measure the outcome. The post test result reflected the improvement in their writing skill and it had found out that this educational website could make the writing lessons become more interesting as they could communicate and learn with peers around the world, particularly native speakers. This positive engagement and finding I think is great and hopeful that the particular application that I want to build that can build this out and prove it at scale would be extremely helpful. 

\item \textbf{Martinez-Romo, J., \& Araujo, L. (2013). Detecting malicious tweets in trending topics using a statistical analysis of language. Expert Systems with Applications, 40(8), 2992-3000.}

This paper is interesting because again one of my goals is to detect malicious language in the application for Pen Pals of K-12 students. Since the students are young it is important to protect their online safety. The authors of this paper look to detect malicious tweets in trending topics using NLP. In this paper the authors present a methodology based on two new aspects: the detection of spam tweets in isolation and without previous information of the user; and the application of a statistical analysis of language to detect spam in trending topics. The authors first collected and labeled a large dataset with 34 K trending topics and 20 million tweets. Then, the authors propose a reduced set of features hardly manipulated by spammers. In addition, the authors developed a machine learning system with some orthogonal features that can be combined with other sets of features with the aim of analyzing emergent characteristics of spam in social networks. They show that they can successful detect harmful tweets. I would look to this paper for motivation on how to detect harmful language and remove it in the application.


\item \textbf{Maxwell, L. R. (2015). Voices of Pen Pals: Exploring the Relationship Between Daily Writing and Writing Development, and Reading Comprehension with Third Grade Students (Doctoral dissertation, Ohio University).}

I picked out this paper since the application that I want to build is essentially an electronic application to allow children to write to each other. This is very similar to the concept of "Pen Pals" so I wanted to look at the research around the effects of Pen Pal programs. This paper investigated a quasi-experimental study that investigated the impact that daily writing instruction and bi-weekly pen pal correspondence had on third graders writing development and reading comprehension in a Midwest, rural elementary school. The treatment group participated in a 12-week writing intervention program in which they exchanged letters with second-grade pen pals on a bi-weekly basis. Letters were informative, expressive, and autonomous, as they were based on daily graphic organizers students completed, on which they wrote about school-related subjects of their choice. The control group did not participate in the writing intervention program, as they received their typical writing instruction. Both groups' reading comprehension scores were assessed a baseline measurement, which was administered prior to the writing treatment, and measurement period measurement, which was administered after the writing treatment had concluded. The authors found that students in the treatment group experienced higher develop in text structure written expression, and audience awareness. This shows the powerful effects of Pen Pal Program. The goal would be to add to this and making a program like this scalable by offering an online application.

\item \textbf{Riess, H. (2016). EMPATHY MATTERS: STUDY SHOWS THAT TEACHING EMPATHY CAN IMPROVE PATIENT SATISFACTION. Iowa medicine: journal of the Iowa Medical Society, 106(1), 13-13.}

This is another paper that looks at the effect of teaching empathy in the context of medical professions. Again, while this is not exactly the context of my application, which focuses on K-12 students, the lessons learned from this are applicable. Empathy matters. This article speaks to how physicians are currently facing tremendous pressures in terms of number of patients they are expected to see in a short amount of time. However, empathy is still important and can improve health outcomes. Similarly, for students today in the modern world there are so many things going on in students' lives. Sports clubs, academics, friend groups, social media all make it so it's hard for students to focus on one thing. That's why this application is so important because we need to build empathy for students so that it improves decision making and they have appreciation for the world. 

\item \textbf{Roberts, G. L., \& McCurdy, M. (2017). Impact of College Pen Pal Program on Fourth Grade Writing Attitudes and Skills.}

This paper is part of the research category around the effect of Pen Pal programs on educational learning for K-12 students. The authors cite a different paper Chohan (2010) that attitudes towards writing in early elementary school students fluctuate as students advance through the writing curriculum from year to year. Research suggests pen pal writing exercises provide a more authentic purpose for writing and may improve writing attitudes for students. The authors introduce a pen pal program between graduate students at the University of Tennessee and fourth graders at an East Tennessee elementary school to examine how attitudes towards writing and overall writing skills would change for participating students. This is interesting to me because it provides a basis that the penpals don't actually need to be peers. It could be a good thing to connect K-12 students with more mature audiences to further build empathy. 

A baseline and post-treatment survey was done after the three-month study period where fourth graders and graduate students corresponded. The results were actually surprisingly mixed. In fact, the authors find that the Pen Pal program was associated with negative emotions towards writing overall and no effect on words used. This is important study to acknowledge because not every Pen Pal program works. One needs to be careful about how we operationalize the program. In this case, it's possible that the gap between 4th graders and graduate students was too big, so it was impossible for it to be an informative conversation. This informs the product that I want to build since match-making will be important to figure how who to connect to who.

\item \textbf{Shandomo, H. M. (2009). Getting to Knew You: Cross-Cultural Pen Pals Expand Children's World View. Childhood education, 85(3), 154-159.}

This paper is another paper that investigates the effect of Pen pal intervention on outcomes. In this particular paper, Shandomo (2009) study was conducted across cultures with the
intention of expanding children’s views of the world and increasing motivation to learn overall. This is especially interesting to me since one of the goals that I want to get out of the application that I want to build is whether pen-pal programs across the world can expand world views and help students gain perspective. Second graders were partnered with students in Zambia, and one set of letters were exchanged. Through reflective journals, the classroom teacher observed the progression of students who previously struggled with writing becoming increasingly interested in and motivated to write. However, the paper struggled since there was no quantitative data regarding improvement in writing skills. Thus, the results of this study are unable to be generalized or widely accepted. However, this is still important for me since it builds intuition even if there isn't hard quantitative support either way.

\item \textbf{Teding van Berkhout, E., \& Malouff, J. M. (2016). The efficacy of empathy training: A meta-analysis of randomized controlled trials. Journal of counseling psychology, 63(1), 32.}

I picked out this article again because I am interesting in how we can teach empathy in the application that I want to build to facilitate pen-pal for K-12 students. This paper I found interesting because of it's meta analysis on randomized controlled trials to look at the efficacy of empathy training. I learned a lot from this article on the quantitative outcomes and how we can actually measure empathy. The meta-analysis included 18 randomized controlled trials of empathy training with a total of 1,018 participants. The survey study found four factors most indicative of success. First, training health professionals and university students rather than other types of individuals. Second, compensating trainees for their participation. Third, using empathy measures that focus exclusively on assessing understanding the emotions of others, feeling those emotions, or commenting accurately on the emotions. Fourth, using objective measures rather than self-report measures. This has significant impact on how I would build and measure success of the application. Especially (3) and (4) signal to me to investigate the quantitative ways I should explore to actually measure improvement of empathy. 

\item \textbf{Thompson McMillon, G. M. (2009). Pen pals without borders: A cultural exchange of teaching and learning. Education and urban society, 42(1), 119-135.}

I picked out this paper also for it's potential to inform the effect of pen-pals on educational outcomes. In this paper, the authors note that when students are from different cultures, dissonance can occur. This is exactly the problem that I'm looking to solve by using the dissonance to expose students to different cultures to allow for more effective literacy teaching and learning. It is imperative that teacher education programs develop creative, effective ways to prepare the teaching population to meet the needs of a diverse student population. This article reports the findings of a pen pal cultural exchange project between 40 predominantly White, female,  teachers in an elementary reading methods course, and 26 predominantly Black, fourth graders in an urban elementary school. Thus, this paper is really relevant to me because it's exactly the kind of thing I would want to do in my app, connect students from different backgrounds together. The study analyzes 336 letters (154 children letters and 182 adult letters) to identity overarching themes. The content of the letters are analyzed using discourse analysis. The three most frequently found themes are shared experiences, overcoming adversities, and cultural practices and experiences. Thus, this paper shows that when connecting different students together, students will naturally talk about their communities and share their stories. 

\item \textbf{Ziff, K., Ivers, N., \& Hutton, K. (2017). “There’s Beauty in Brokenness”: Teaching Empathy Through Dialogue with Art. Journal of Creativity in Mental Health, 12(2), 249-261.}

I picked out this article again as part of research in how we teach empathy. Empathy is very critical in all aspects of life and is a skill I am hoping that the application I build is part of that. This article presents a classroom exercise developed to increase students' empathy. The exercise features imaginative dialogue by members of a counselor education beginning skills class with art works in an exhibit curated by a museum educator. Presented are the details of the teaching exercise, student and faculty reflections on the exercise, and suggestions for further research. This article is extremely interesting because I think it's a novel way to teach empathy. Based on this article I am wondering if not only do I want to motivate pen-pals with writing prompts but also perhaps other pictures and references to art work!

\end{enumerate}

\end{document}
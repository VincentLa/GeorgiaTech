\documentclass[12pt, final]{article}
\usepackage{fullpage,amsthm,amsfonts,amsmath}
\usepackage{enumerate}
\usepackage[margin=1in]{geometry}
\usepackage{color}
\usepackage{epsfig}
\usepackage{epstopdf}
\usepackage{datatool}
\usepackage{array}
\usepackage{tabu}
\usepackage{amsmath}
\usepackage{caption}
\usepackage{float}
\usepackage{tabularx, booktabs}
\usepackage{pdfpages}
\usepackage{standalone}
\usepackage{hyperref}
\usepackage{subfigure}
\usepackage{graphicx}
\hypersetup{colorlinks = true, urlcolor = blue, filecolor = blue, linkcolor = blue}
\floatstyle{plaintop}
\restylefloat{table}
\renewcommand{\thefootnote}{$\star$} 

\begin{document}
\title{Exploring Educational Technology (Spring 2019) Assignment 2}

\date{\today}

%%
\renewcommand{\thefootnote}{$\dag$}
%%
\author{Vincent La\footnote{vincent.la@gatech.edu}}

\maketitle

\begin{abstract}
This assignment surveys the expanse of educational technology research community. We will look at a variety of types of sources to survey how the field has targeted different audiences, different specific content areas, and different technologies. We will focus our discussion around seven different categories: Technology, Audience, Content, Sociotechnical Issue, Theory, Medium, and Methodology. 
\end{abstract} 

\newpage
%%%%%
\renewcommand{\thefootnote}{\number\value{footnote}} 
%%%%%
\section{Source 1: Can Online Delivery increase Access to Education} \label{Source 1}
For my first source, I cite a research paper Goodman (2019) \cite{Goodman} as it researches whether or not Online Delivery increases Access to Education. Furthermore, it uses data from OMSCS which I thought was really interesting!
\\
\\
The paper was originally distributed as a working paper via National Bureau of Economic Research and was eventually published in the Journal of Labor Economics. The paper is the first that investigates whether online education actually affects the number of people pursuing education. Using the OMSCS program, the paper uses a Regression Discontinuity approach with exploits an admissions threshold that shows that access to the online option actually does increase enrollment into education. In conclusion, it estimates that OMSCS will boost the annual production of US CS Masters Degrees by about seven percent!

\subsection{Technology}
\textit{What technology or technologies does the source use? For example: chatbots, data analytics, virtual reality, wearable devices, etc.}
\\
\\
The technology that the source uses is Data Analytics/Statistics primarily using Regression discontinuity technique to try to estimate causal effects of the program on number of people pursuing education. Specifically, in the very first cohort, the program's  admission officer read applications in descending order of undergraduate GPA until he had identified about 500 applicants to which immediate admission was offered. As a result, such offers were made only to those with a GPA of at least 3.26, a threshold that was arbitrary and unknown to applicants. The officer eventually read all of the applications and some of those below the threshold were offered deferred admission. A regression discontinuity design shows this admissions process created at the threshold a roughly 20 percentage point difference in the probability
of admission to the online program. The paper then used National Student Clearning House data to compare enrollment outcomes for those applications just under vs just over the threshold. 

\subsection{Audience}
\textit{Who is the target audience for the source? For example: K-12 students, working professionals, informal learners, etc.}
\\
\\
Audience is for the academic (Economics) research community, program administrators, policy makers.

\subsection{Content}
\textit{What content is being taught by the source? For example: computer science, math, writing, geography, etc.}
\\
\\
Labor Economics

\subsection{Sociotechnical Issue}
\textit{What issue is the source addressing? For example: access to technology, gender disparities in technology use, use of technology for social good or social evil, etc.}
\\
\\
Access to Education, especially for mid-career adults who may not have good access to affordable but high quality education.

\subsection{Theory}
\textit{What learning theory is the source leveraging or investigating? For example: communities of practice, social learning, spaced repetition, etc.}
\\
\\
Communities of practice as it is studying the overall mechanism of how Online Masters programs (specifically OMSCS) affects enrollment into academic institution.

\subsection{Medium}
\textit{What medium is the source using to deliver its idea? For example: traditional classrooms, MOOCs, informal museum displays, etc.}
\\
\\
Research Paper published in Academic Journal.

\subsection{Methodology}
\textit{How does the source investigate its idea? For example: controlled experiments, surveys and interviews, naturalistic observation, etc.}
\\
\\
Quasi-experimental using Regression discontinuity design on observational data due to exploiting the fact that the very first OMSCS cohort had a artificial GPA cutoff.

\section{Source 2: Evaluating Teachers: The important role of value-added} \label{Source 2}

The next source we will discuss is Glazerman (2010) \cite{Glazerman}. This paper was published in the Brown Center on Education Policy at Brookings. This source is a survey study on ``Value-Added" methodologies to measure effectiveness of teachers (generally in the K-12 range). In summary, ``Value-Added" refers to the measurement of the evaluation of teachers based on the contribution they make to the learning of their students. However, it has been controversial since many Value-Added methodologies use standardized test scores as the base data. Standardized test scores may not fully capture the nuance of the true contributions that a teacher makes on a student. Furthermore, it is very difficult to separate out value added of the teacher vs student characteristics. In particular, naively looking at test scores of schools in worse neighborhoods correlated with worse socioeconomic characteristics would imply systems level problems, not teacher-specific.
\\
\\
That being said, this paper discusses the merits of ``Value-Added" methodologies. Current methods are clearly not working. In many school districts where the teachers are evaluated on a "binary" metric (e.g. Satisfactory vs not) most teachers are given "satisfactory" evaluations with no differentiation between teachers who contribute a lot and are high performing vs low-performing teachers. In addition, ``Value-Added" methodologies, although imperfect have been shown to have definitely non-zero relationship with educational outcomes. 

\subsection{Technology}
\textit{What technology or technologies does the source use? For example: chatbots, data analytics, virtual reality, wearable devices, etc.}
\\
\\
This source uses data analytics, regression methods using ``Value-Added" methodologies. For further technical discussion, the seminal paper on applying ``Value-Added" methodologies within education is McCaffrey (2004) \cite{McCaffrey}.

\subsection{Audience}
\textit{Who is the target audience for the source? For example: K-12 students, working professionals, informal learners, etc.}
\\
\\
K-12 students and professionals, policy makers, teachers.

\subsection{Content}
\textit{What content is being taught by the source? For example: computer science, math, writing, geography, etc.}
\\
\\
Education Statistics, Economics

\subsection{Sociotechnical Issue}
\textit{What issue is the source addressing? For example: access to technology, gender disparities in technology use, use of technology for social good or social evil, etc.}
\\
\\
How do we properly evaluate teachers to better differentiate and improve our education system

\subsection{Theory}
\textit{What learning theory is the source leveraging or investigating? For example: communities of practice, social learning, spaced repetition, etc.}
\\
\\
Communities of practice as the source and the results from the source are about who should constitute as our children's educators.

\subsection{Medium}
\textit{What medium is the source using to deliver its idea? For example: traditional classrooms, MOOCs, informal museum displays, etc.}
\\
\\
Research Paper

\subsection{Methodology}
\textit{How does the source investigate its idea? For example: controlled experiments, surveys and interviews, naturalistic observation, etc.}
\\
\\
Value added regression methodology. Again see McCaffrey (2004) \cite{McCaffrey} for the seminal paper on using ``Value Added" in education.


\section{Source 3: Leveraging Educational Technology to Improve the Quality of Civil Discourse} \label{Source 3}
The above two sources were sourced from Economics and Education sources (although different topics -- College vs K-12 and studying very different topics). In the next couple of sources, I'll talk more about AI in Education\footnote{Note that many interesting papers were highlighted in this blog post: https://www.veletsianos.com/2018/06/27/10-interesting-papers-in-the-proceedings-of-the-artificial-intelligence-in-education-2018-conference-aied18/}. 
\\
\\
The next source I review is Diana (2018) \cite{Diana}. This paper was initially presented in the Artificial Intelligence in Education 2018 Conference. The article is about how in the current world, there is lots of "Fake News" out there. That there are news and articles shared that share information, but the veracity of the source is often very hard to discern which is real vs which is not. In addition, it is often much easier to spew "fake news" then it is to prove that it is false.
\\
\\
This paper offers a potential solution: to use AI to build a process to classify what is "Fake news" vs real. This potentially gives a chance to classify such news at scale.

\subsection{Technology}
\textit{What technology or technologies does the source use? For example: chatbots, data analytics, virtual reality, wearable devices, etc.}
\\
\\
The technologies that the source uses is 3-fold. First, it's an online tutorial system to educate people for teaching logical fallacy identification. Second, the system has an application that allows crowd workers to identify potentially fallacious arguments (so almost like a mechanical turk). And then finally, it builds a model using the labeled dataset to classify news articles into fake vs real.

\subsection{Audience}
\textit{Who is the target audience for the source? For example: K-12 students, working professionals, informal learners, etc.}
\\
\\
Professionals, informal learners, policy makers, general citizens of the world

\subsection{Content}
\textit{What content is being taught by the source? For example: computer science, math, writing, geography, etc.}
\\
\\
Artificial Intelligence, computer science, news, politics

\subsection{Sociotechnical Issue}
\textit{What issue is the source addressing? For example: access to technology, gender disparities in technology use, use of technology for social good or social evil, etc.}
\\
\\
How to ensure accurate dissemination of information; technology for social good

\subsection{Theory}
\textit{What learning theory is the source leveraging or investigating? For example: communities of practice, social learning, spaced repetition, etc.}
\\
\\
Social Learning as it is engaging a crowd of people to learn from a platform and potentially teach each other

\subsection{Medium}
\textit{What medium is the source using to deliver its idea? For example: traditional classrooms, MOOCs, informal museum displays, etc.}
\\
\\
Online Application; Cognitive Tutors

\subsection{Methodology}
\textit{How does the source investigate its idea? For example: controlled experiments, surveys and interviews, naturalistic observation, etc.}
\\
\\
Builds an application and system that, as mentioned previously takes a three-fold strategy. First, it's an online tutorial system to educate people for teaching logical fallacy identification. Second, the system has an application that allows crowd workers to identify potentially fallacious arguments (so almost like a mechanical turk). And then finally, it builds a model using the labeled dataset to classify news articles into fake vs real.

\section{Source 4: Towards Combined Network and Text Analytics of Student Discourse in Online Discussions} \label{Source 4}
The next paper I discuss is Ferreira (2018) \cite{Ferreira}. This paper was also discussed in the International Conference on Artificial Intelligence in Education in 2018. It describes how to use Natural Language Processing and Topic Modeling to evaluate students' use of asynchronous discussions (e.g. on an Online Platform like Piazza) in online learning environments. What's interesting is how the paper examines students' cognitive development across different course topics. 

\subsection{Technology}
\textit{What technology or technologies does the source use? For example: chatbots, data analytics, virtual reality, wearable devices, etc.}
\\
\\
Epistemic network analysis, Natural Language Processing, Topic Modeling

\subsection{Audience}
\textit{Who is the target audience for the source? For example: K-12 students, working professionals, informal learners, etc.}
\\
\\
Researchers, Educators

\subsection{Content}
\textit{What content is being taught by the source? For example: computer science, math, writing, geography, etc.}
\\
\\
Computer Science, Artificial Intelligence, NLP

\subsection{Sociotechnical Issue}
\textit{What issue is the source addressing? For example: access to technology, gender disparities in technology use, use of technology for social good or social evil, etc.}
\\
\\
Use of technology for social good/education

\subsection{Theory}
\textit{What learning theory is the source leveraging or investigating? For example: communities of practice, social learning, spaced repetition, etc.}
\\
\\
Community of Inquiry Model, Social Learning -- (studying online community learning threads)

\subsection{Medium}
\textit{What medium is the source using to deliver its idea? For example: traditional classrooms, MOOCs, informal museum displays, etc.}
\\
\\
Academic Research Paper

\subsection{Methodology}
\textit{How does the source investigate its idea? For example: controlled experiments, surveys and interviews, naturalistic observation, etc.}
\\
\\
Combination of network-based analysis and natural-language processing (NLP) techniques used to provide more detailed insights into student learning in asynchronous online discussions

\section{Source 5: }
\label{Source 5}

\subsection{Technology}
\textit{What technology or technologies does the source use? For example: chatbots, data analytics, virtual reality, wearable devices, etc.}

\subsection{Audience}
\textit{Who is the target audience for the source? For example: K-12 students, working professionals, informal learners, etc.}

\subsection{Content}
\textit{What content is being taught by the source? For example: computer science, math, writing, geography, etc.}

\subsection{Sociotechnical Issue}
\textit{What issue is the source addressing? For example: access to technology, gender disparities in technology use, use of technology for social good or social evil, etc.}

\subsection{Theory}
\textit{What learning theory is the source leveraging or investigating? For example: communities of practice, social learning, spaced repetition, etc.}

\subsection{Medium}
\textit{What medium is the source using to deliver its idea? For example: traditional classrooms, MOOCs, informal museum displays, etc.}

\subsection{Methodology}
\textit{How does the source investigate its idea? For example: controlled experiments, surveys and interviews, naturalistic observation, etc.}

\section{Source 6: Learning to cope: Voluntary financial education and loan performance during a housing crisis}
\label{Source 6}

Next, I cite \cite{Agarwal} which looks at how voluntary financial education can affect loan default rates during a housing crises.

\subsection{Technology}
\textit{What technology or technologies does the source use? For example: chatbots, data analytics, virtual reality, wearable devices, etc.}
\\
\\
Econometric Analysis

\subsection{Audience}
\textit{Who is the target audience for the source? For example: K-12 students, working professionals, informal learners, etc.}
\\
\\
Working Professionals, Researchers, also any Mortgage Owner

\subsection{Content}
\textit{What content is being taught by the source? For example: computer science, math, writing, geography, etc.}
\\
\\
Economics/Household finances

\subsection{Sociotechnical Issue}
\textit{What issue is the source addressing? For example: access to technology, gender disparities in technology use, use of technology for social good or social evil, etc.}
\\
\\
Use of technology to evaluate intervention for social good

\subsection{Theory}
\textit{What learning theory is the source leveraging or investigating? For example: communities of practice, social learning, spaced repetition, etc.}
\\
\\
Social Learning

\subsection{Medium}
\textit{What medium is the source using to deliver its idea? For example: traditional classrooms, MOOCs, informal museum displays, etc.}
\\
\\
Research Paper

\subsection{Methodology}
\textit{How does the source investigate its idea? For example: controlled experiments, surveys and interviews, naturalistic observation, etc.}
\\
\\
Econometric analysis using data from Indianapolis Neighborhood Housing Partnership Data

\section{Source 7: Predictive modeling to forecast student outcomes and drive effective interventions in online community college courses}
\label{Source 7}

Next, I cite \cite{Smith} which looks at predictive modeling to forecast student outcomes and drive effective interventions

\subsection{Technology}
\textit{What technology or technologies does the source use? For example: chatbots, data analytics, virtual reality, wearable devices, etc.}
\\
\\
Machine Learning/ Predictive Modeling

\subsection{Audience}
\textit{Who is the target audience for the source? For example: K-12 students, working professionals, informal learners, etc.}
\\
\\
Working Professionals, Researchers, Students

\subsection{Content}
\textit{What content is being taught by the source? For example: computer science, math, writing, geography, etc.}
\\
\\
Computer Science

\subsection{Sociotechnical Issue}
\textit{What issue is the source addressing? For example: access to technology, gender disparities in technology use, use of technology for social good or social evil, etc.}
\\
\\
Use of technology for social good

\subsection{Theory}
\textit{What learning theory is the source leveraging or investigating? For example: communities of practice, social learning, spaced repetition, etc.}
\\
\\
Social Learning

\subsection{Medium}
\textit{What medium is the source using to deliver its idea? For example: traditional classrooms, MOOCs, informal museum displays, etc.}
\\
\\
Online College Classes

\subsection{Methodology}
\textit{How does the source investigate its idea? For example: controlled experiments, surveys and interviews, naturalistic observation, etc.}
\\
\\
Predictive Modeling on student characteristics to identify at-risk students to provide early interventions

\section{Source 8: Opportunities for natural language processing research in education}
\label{Source 8}

Next, I cite \cite{Burstein}.

\subsection{Technology}
\textit{What technology or technologies does the source use? For example: chatbots, data analytics, virtual reality, wearable devices, etc.}
\\
\\
Natural Language Processing

\subsection{Audience}
\textit{Who is the target audience for the source? For example: K-12 students, working professionals, informal learners, etc.}
\\
\\
Researchers

\subsection{Content}
\textit{What content is being taught by the source? For example: computer science, math, writing, geography, etc.}
\\
\\
Computer Science/NLP in Education

\subsection{Sociotechnical Issue}
\textit{What issue is the source addressing? For example: access to technology, gender disparities in technology use, use of technology for social good or social evil, etc.}
\\
\\
Use of Technology for Education

\subsection{Theory}
\textit{What learning theory is the source leveraging or investigating? For example: communities of practice, social learning, spaced repetition, etc.}
\\
\\
Social Learning

\subsection{Medium}
\textit{What medium is the source using to deliver its idea? For example: traditional classrooms, MOOCs, informal museum displays, etc.}
\\
\\
Textbook/Traditional Classrooms/MOOCs -- pretty general

\subsection{Methodology}
\textit{How does the source investigate its idea? For example: controlled experiments, surveys and interviews, naturalistic observation, etc.}
\\
\\
Natural Language Processing to develop applications

\section{Source 9: The causal effect of education on health: Evidence from the United Kingdom}
\label{Source 9}

Next, I cite \cite{Siles} who examines causal effect of Education on Health; evidence from the United Kingdom

\subsection{Technology}
\textit{What technology or technologies does the source use? For example: chatbots, data analytics, virtual reality, wearable devices, etc.}
\\
\\
Econometric Analysis

\subsection{Audience}
\textit{Who is the target audience for the source? For example: K-12 students, working professionals, informal learners, etc.}
\\
\\
Working Professionals, informal learners

\subsection{Content}
\textit{What content is being taught by the source? For example: computer science, math, writing, geography, etc.}
\\
\\
Economics, Education, Health

\subsection{Sociotechnical Issue}
\textit{What issue is the source addressing? For example: access to technology, gender disparities in technology use, use of technology for social good or social evil, etc.}
\\
\\
Effect of Education of Health

\subsection{Theory}
\textit{What learning theory is the source leveraging or investigating? For example: communities of practice, social learning, spaced repetition, etc.}
\\
\\
Not Applicable

\subsection{Medium}
\textit{What medium is the source using to deliver its idea? For example: traditional classrooms, MOOCs, informal museum displays, etc.}
\\
\\
Research Paper

\subsection{Methodology}
\textit{How does the source investigate its idea? For example: controlled experiments, surveys and interviews, naturalistic observation, etc.}
\\
\\
Econometric Analysis

\section{Source 10: Education and health: evaluating theories and evidence}
\label{Source 10}

Next, I cite \cite{Cutler} who looks at the causal effect of education on health

\subsection{Technology}
\textit{What technology or technologies does the source use? For example: chatbots, data analytics, virtual reality, wearable devices, etc.}
\\
\\
Econometrics Research

\subsection{Audience}
\textit{Who is the target audience for the source? For example: K-12 students, working professionals, informal learners, etc.}
\\
\\
Informal Learners, Academics, Researchers

\subsection{Content}
\textit{What content is being taught by the source? For example: computer science, math, writing, geography, etc.}
\\
\\
Economics, Health, Education

\subsection{Sociotechnical Issue}
\textit{What issue is the source addressing? For example: access to technology, gender disparities in technology use, use of technology for social good or social evil, etc.}
\\
\\
Use of technology for social good

\subsection{Theory}
\textit{What learning theory is the source leveraging or investigating? For example: communities of practice, social learning, spaced repetition, etc.}
\\
\\
Not Applicable

\subsection{Medium}
\textit{What medium is the source using to deliver its idea? For example: traditional classrooms, MOOCs, informal museum displays, etc.}
\\
\\
Research Paper

\subsection{Methodology}
\textit{How does the source investigate its idea? For example: controlled experiments, surveys and interviews, naturalistic observation, etc.}
\\
\\
Econometric Analysis

\section{Source 11: Who falls for fake news? The roles of bullshit receptivity, overclaiming, familiarity, and analytic thinking.}
\label{Source 11}

Next, I cite \cite{Pennycook} who offers a Social Science perspective on the archetype of someone who believes in fake news.

\subsection{Technology}
\textit{What technology or technologies does the source use? For example: chatbots, data analytics, virtual reality, wearable devices, etc.}
\\
\\
Experimental Analysis

\subsection{Audience}
\textit{Who is the target audience for the source? For example: K-12 students, working professionals, informal learners, etc.}
\\
\\
Informal Learners

\subsection{Content}
\textit{What content is being taught by the source? For example: computer science, math, writing, geography, etc.}
\\
\\
Statistics/Social Science

\subsection{Sociotechnical Issue}
\textit{What issue is the source addressing? For example: access to technology, gender disparities in technology use, use of technology for social good or social evil, etc.}
\\
\\
Use of technology for social good

\subsection{Theory}
\textit{What learning theory is the source leveraging or investigating? For example: communities of practice, social learning, spaced repetition, etc.}
\\
\\
Social Learning

\subsection{Medium}
\textit{What medium is the source using to deliver its idea? For example: traditional classrooms, MOOCs, informal museum displays, etc.}
\\
\\
Research

\subsection{Methodology}
\textit{How does the source investigate its idea? For example: controlled experiments, surveys and interviews, naturalistic observation, etc.}
\\
\\
Experiments using real human subjects

\section{Source 12: Truth of varying shades: Analyzing language in fake news and political fact-checking}
\label{Source 12}

For this source, I cite \cite{Rashkin} which looks at analyzing language in fake news and political fact checking.

\subsection{Technology}
\textit{What technology or technologies does the source use? For example: chatbots, data analytics, virtual reality, wearable devices, etc.}
\\
\\
NLP and Analytics

\subsection{Audience}
\textit{Who is the target audience for the source? For example: K-12 students, working professionals, informal learners, etc.}
\\
\\
Informal Learners

\subsection{Content}
\textit{What content is being taught by the source? For example: computer science, math, writing, geography, etc.}
\\
\\
Computer Science, Politics

\subsection{Sociotechnical Issue}
\textit{What issue is the source addressing? For example: access to technology, gender disparities in technology use, use of technology for social good or social evil, etc.}
\\
\\
Use of technology for social good (analyzing type of language in fake news)

\subsection{Theory}
\textit{What learning theory is the source leveraging or investigating? For example: communities of practice, social learning, spaced repetition, etc.}
\\
\\
Social Learning

\subsection{Medium}
\textit{What medium is the source using to deliver its idea? For example: traditional classrooms, MOOCs, informal museum displays, etc.}
\\
\\
Research Paper

\subsection{Methodology}
\textit{How does the source investigate its idea? For example: controlled experiments, surveys and interviews, naturalistic observation, etc.}
\\
\\
Conducted Natural Language Processing and Analytics to find language in Fake news vs real news.

\section{Source 13: ``Liar, liar pants on fire": A new benchmark dataset for fake news detection}
\label{Source 13}

Next, I cite \cite{Wang} who provides a new brenchmark dataset for fake news detection.

\subsection{Technology}
\textit{What technology or technologies does the source use? For example: chatbots, data analytics, virtual reality, wearable devices, etc.}
\\
\\
Deep Learning on PolitiFact.com news articles classified into fake news or real.

\subsection{Audience}
\textit{Who is the target audience for the source? For example: K-12 students, working professionals, informal learners, etc.}
\\
\\
Working Professionals, Researchers on Fake News

\subsection{Content}
\textit{What content is being taught by the source? For example: computer science, math, writing, geography, etc.}
\\
\\
Computer Science, Politics

\subsection{Sociotechnical Issue}
\textit{What issue is the source addressing? For example: access to technology, gender disparities in technology use, use of technology for social good or social evil, etc.}
\\
\\
Use of technology for social good

\subsection{Theory}
\textit{What learning theory is the source leveraging or investigating? For example: communities of practice, social learning, spaced repetition, etc.}
\\
\\
Not applicable

\subsection{Medium}
\textit{What medium is the source using to deliver its idea? For example: traditional classrooms, MOOCs, informal museum displays, etc.}
\\
\\
Online data set

\subsection{Methodology}
\textit{How does the source investigate its idea? For example: controlled experiments, surveys and interviews, naturalistic observation, etc.}
\\
\\
Provides dataset labeled with fake news using politifacts.com data.

\section{Source 14: Automatic deception detection: Methods for finding fake news}
\label{Source 14}

\subsection{Technology}
\textit{What technology or technologies does the source use? For example: chatbots, data analytics, virtual reality, wearable devices, etc.}
\\
\\
Survey study of machine learning literature and natural language processing to detect fake news

\subsection{Audience}
\textit{Who is the target audience for the source? For example: K-12 students, working professionals, informal learners, etc.}
\\
\\
Informal Learners, general citizens of the world

\subsection{Content}
\textit{What content is being taught by the source? For example: computer science, math, writing, geography, etc.}
\\
\\
Politics, computer sicne

\subsection{Sociotechnical Issue}
\textit{What issue is the source addressing? For example: access to technology, gender disparities in technology use, use of technology for social good or social evil, etc.}
\\
\\
Use of technology for social good (detecting fake news)

\subsection{Theory}
\textit{What learning theory is the source leveraging or investigating? For example: communities of practice, social learning, spaced repetition, etc.}
\\
\\
Social Learning (how fake news affects people)

\subsection{Medium}
\textit{What medium is the source using to deliver its idea? For example: traditional classrooms, MOOCs, informal museum displays, etc.}
\\
\\
Research survey study

\subsection{Methodology}
\textit{How does the source investigate its idea? For example: controlled experiments, surveys and interviews, naturalistic observation, etc.}
\\
\\
Surveys research and provides literature review

\section{Source 15: Social media and fake news in the 2016 election}
\label{Source 15}

Next, I cite \cite{Allcott} around Social Media and fake news in the 2016 election.

\subsection{Technology}
\textit{What technology or technologies does the source use? For example: chatbots, data analytics, virtual reality, wearable devices, etc.}
\\
\\
Economic Modeling

\subsection{Audience}
\textit{Who is the target audience for the source? For example: K-12 students, working professionals, informal learners, etc.}
\\
\\
Informal Learners, American Citizens, Working Professionals

\subsection{Content}
\textit{What content is being taught by the source? For example: computer science, math, writing, geography, etc.}
\\
\\
Politics

\subsection{Sociotechnical Issue}
\textit{What issue is the source addressing? For example: access to technology, gender disparities in technology use, use of technology for social good or social evil, etc.}
\\
\\
Use of technology for social evil (Fake news disseminating false information and affecting politics)

\subsection{Theory}
\textit{What learning theory is the source leveraging or investigating? For example: communities of practice, social learning, spaced repetition, etc.}
\\
\\
Social Learning (how fake news affects society)

\subsection{Medium}
\textit{What medium is the source using to deliver its idea? For example: traditional classrooms, MOOCs, informal museum displays, etc.}
\\
\\
Research paper looking at economic modeling. Found fake news was largely in favor of Trump.

\subsection{Methodology}
\textit{How does the source investigate its idea? For example: controlled experiments, surveys and interviews, naturalistic observation, etc.}

\section{Source 16: Automated Pitch Convergence Improves Learning in a Social, Teachable Robot for Middle School Mathematics.}
\label{Source 16}

Next, I cite \cite{Lubold} on whether pedagogical agents who entrains and speaks more socially can actually encourage more learning.

\subsection{Technology}
\textit{What technology or technologies does the source use? For example: chatbots, data analytics, virtual reality, wearable devices, etc.}
\\
\\
Pedagogical agents/Robots

\subsection{Audience}
\textit{Who is the target audience for the source? For example: K-12 students, working professionals, informal learners, etc.}
\\
\\
K-12 students; School Administrators

\subsection{Content}
\textit{What content is being taught by the source? For example: computer science, math, writing, geography, etc.}
\\
\\
General Education

\subsection{Sociotechnical Issue}
\textit{What issue is the source addressing? For example: access to technology, gender disparities in technology use, use of technology for social good or social evil, etc.}
\\
\\
Use of Technology for social good

\subsection{Theory}
\textit{What learning theory is the source leveraging or investigating? For example: communities of practice, social learning, spaced repetition, etc.}
\\
\\
Social Learning

\subsection{Medium}
\textit{What medium is the source using to deliver its idea? For example: traditional classrooms, MOOCs, informal museum displays, etc.}
\\
\\
Traditional Classrooms

\subsection{Methodology}
\textit{How does the source investigate its idea? For example: controlled experiments, surveys and interviews, naturalistic observation, etc.}
\\
\\
Experiments where it used one type of robot who does more entrainment and social cues and another robot that doesn't. Found the former robot encouraged more learning.

\section{Source 17: Female Robots as Role-Models?-The Influence of Robot Gender and Learning Materials on Learning Success}
\label{Source 17}

Next, I cite \cite{Pfeifer} on the influence of Robot Gender and Learning Materials on Learning Success.

\subsection{Technology}
\textit{What technology or technologies does the source use? For example: chatbots, data analytics, virtual reality, wearable devices, etc.}
\\
\\
AI; Robots

\subsection{Audience}
\textit{Who is the target audience for the source? For example: K-12 students, working professionals, informal learners, etc.}
\\
\\
K-12 Students

\subsection{Content}
\textit{What content is being taught by the source? For example: computer science, math, writing, geography, etc.}
\\
\\
Computer Science

\subsection{Sociotechnical Issue}
\textit{What issue is the source addressing? For example: access to technology, gender disparities in technology use, use of technology for social good or social evil, etc.}
\\
\\
Gender Disparities in Technology Use

\subsection{Theory}
\textit{What learning theory is the source leveraging or investigating? For example: communities of practice, social learning, spaced repetition, etc.}
\\
\\
Social Learning

\subsection{Medium}
\textit{What medium is the source using to deliver its idea? For example: traditional classrooms, MOOCs, informal museum displays, etc.}
\\
\\
Traditional Classrooms

\subsection{Methodology}
\textit{How does the source investigate its idea? For example: controlled experiments, surveys and interviews, naturalistic observation, etc.}
\\
\\
Built a robot and had it speak in a female voice, studied effect on Learning outcomes.

\section{Source 18: Time Series Model for Predicting Dropout in Massive Open Online Courses}
\label{Source 18}

Next, I cite \cite{Tang} which uses Time Series Model for predicting Dropout in Massive Open Online Courses

\subsection{Technology}
\textit{What technology or technologies does the source use? For example: chatbots, data analytics, virtual reality, wearable devices, etc.}
\\
\\
Predictive Analytics

\subsection{Audience}
\textit{Who is the target audience for the source? For example: K-12 students, working professionals, informal learners, etc.}
\\
\\
MOOC Administrators, general MOOC students

\subsection{Content}
\textit{What content is being taught by the source? For example: computer science, math, writing, geography, etc.}
\\
\\
Predictive Analytics

\subsection{Sociotechnical Issue}
\textit{What issue is the source addressing? For example: access to technology, gender disparities in technology use, use of technology for social good or social evil, etc.}
\\
\\
Use of technology for social good; predicting dropout in Massive Open Online Courses to potentially improve content

\subsection{Theory}
\textit{What learning theory is the source leveraging or investigating? For example: communities of practice, social learning, spaced repetition, etc.}
\\
\\
Not Applicable

\subsection{Medium}
\textit{What medium is the source using to deliver its idea? For example: traditional classrooms, MOOCs, informal museum displays, etc.}
\\
\\
MOOC

\subsection{Methodology}
\textit{How does the source investigate its idea? For example: controlled experiments, surveys and interviews, naturalistic observation, etc.}
\\
\\
Uses predictive modeling to predict whether a student will drop a course in a MOOC or not. 

\section{Source 19: Quantifying Classroom Instructor Dynamics with Computer Vision.}
\label{Source 19}

Next, I cite \cite{Bosch} which examines quantifying Classroom Instructor Dynamics with Computer Vision.

\subsection{Technology}
\textit{What technology or technologies does the source use? For example: chatbots, data analytics, virtual reality, wearable devices, etc.}
\\
\\
Deep Learning, Computer Vision

\subsection{Audience}
\textit{Who is the target audience for the source? For example: K-12 students, working professionals, informal learners, etc.}
\\
\\
K-12 Students, Teachers

\subsection{Content}
\textit{What content is being taught by the source? For example: computer science, math, writing, geography, etc.}
\\
\\
General educational methods

\subsection{Sociotechnical Issue}
\textit{What issue is the source addressing? For example: access to technology, gender disparities in technology use, use of technology for social good or social evil, etc.}
\\
\\
Use of technology to enhance classroom teaching

\subsection{Theory}
\textit{What learning theory is the source leveraging or investigating? For example: communities of practice, social learning, spaced repetition, etc.}
\\
\\
Social Learning

\subsection{Medium}
\textit{What medium is the source using to deliver its idea? For example: traditional classrooms, MOOCs, informal museum displays, etc.}
\\
\\
Traditional classrooms

\subsection{Methodology}
\textit{How does the source investigate its idea? For example: controlled experiments, surveys and interviews, naturalistic observation, etc.}
\\
\\
Uses deep learning to examine interactions in classroom setting and quantifies the classroom dynamics.

\section{Source 20: Communication at Scale in a MOOC Using Predictive Engagement Analytics}
\label{Source 20}

Finally, I discuss \cite{Le} which uses predictive modeling on Engagement Analytics to enhance learnings in MOOC.

\subsection{Technology}
\textit{What technology or technologies does the source use? For example: chatbots, data analytics, virtual reality, wearable devices, etc.}
\\
\\
MOOC Platform with predictive modeling of engagement analysis

\subsection{Audience}
\textit{Who is the target audience for the source? For example: K-12 students, working professionals, informal learners, etc.}
\\
\\
Working Professionals/MOOC Learners

\subsection{Content}
\textit{What content is being taught by the source? For example: computer science, math, writing, geography, etc.}
\\
\\
Computer Science/General MOOC Subjects

\subsection{Sociotechnical Issue}
\textit{What issue is the source addressing? For example: access to technology, gender disparities in technology use, use of technology for social good or social evil, etc.}
\\
\\
Technology for educational purposes

\subsection{Theory}
\textit{What learning theory is the source leveraging or investigating? For example: communities of practice, social learning, spaced repetition, etc.}
\\
\\
Social Learning

\subsection{Medium}
\textit{What medium is the source using to deliver its idea? For example: traditional classrooms, MOOCs, informal museum displays, etc.}
\\
\\
MOOCs

\subsection{Methodology}
\textit{How does the source investigate its idea? For example: controlled experiments, surveys and interviews, naturalistic observation, etc.}
\\
\\
Predictive modeling to target learners who are predicted to finish but not pass the class.

  \begin{thebibliography}{1}
  \bibitem{Agarwal} Agarwal, S., Amromin, G., Ben-David, I., Chomsisengphet, S., \& Evanoff, D. D. (2010). Learning to cope: Voluntary financial education and loan performance during a housing crisis. American Economic Review, 100(2), 495-500.
  \bibitem{Allcott} Allcott, H., \& Gentzkow, M. (2017). Social media and fake news in the 2016 election. Journal of Economic Perspectives, 31(2), 211-36.
  \bibitem{Bosch} Bosch, N., Mills, C., Wammes, J. D., \& Smilek, D. (2018, June). Quantifying Classroom Instructor Dynamics with Computer Vision. In International Conference on Artificial Intelligence in Education (pp. 30-42). Springer, Cham.
  \bibitem{Burstein} Burstein, J. (2009, March). Opportunities for natural language processing research in education. In International Conference on Intelligent Text Processing and Computational Linguistics (pp. 6-27). Springer, Berlin, Heidelberg.
  \bibitem{Conroy} Conroy, N. J., Rubin, V. L., \& Chen, Y. (2015, November). Automatic deception detection: Methods for finding fake news. In Proceedings of the 78th ASIS\&T Annual Meeting: Information Science with Impact: Research in and for the Community (p. 82). American Society for Information Science.
  \bibitem{Cutler} Cutler, D. M., \& Lleras-Muney, A. (2006). Education and health: evaluating theories and evidence (No. w12352). National bureau of economic research.
   \bibitem{Diana} Diana, N. (2018, June). Leveraging Educational Technology to Improve the Quality of Civil Discourse. In International Conference on Artificial Intelligence in Education (pp. 517-520). Springer, Cham.
   \bibitem{Ferreira} Ferreira, R., Kovanovic, V., Gasevic, D., Rolim, V. (2018, June). Towards Combined Network and Text Analytics of Student Discourse in Online Discussions. In International Conference on Artificial Intelligence in Education (pp. 111-126). Springer, Cham.
   \bibitem{Le} Le, C. V., Pardos, Z. A., Meyer, S. D., \& Thorp, R. (2018, June). Communication at Scale in a MOOC Using Predictive Engagement Analytics. In International Conference on Artificial Intelligence in Education (pp. 239-252). Springer, Cham.
    \bibitem{Glazerman} Glazerman, S., Loeb, S., Goldhaber, D., Raudenbush, D., Staiger, D., \& Whitehurst, G.J. (2010). Evaluating teachers: The important role of value-added. The Brookings Brown Center.
    \bibitem{Goodman} Goodman, J., Melkers, J., \& Pallais, A. (2019). Can online delivery increase access to education?. Journal of Labor Economics, 37(1), 000-000.
    \bibitem{Lubold} Lubold, N., Walker, E., Pon-Barry, H., \& Ogan, A. (2018, June). Automated Pitch Convergence Improves Learning in a Social, Teachable Robot for Middle School Mathematics. In International Conference on Artificial Intelligence in Education (pp. 282-296). Springer, Cham.
    \bibitem{McCaffrey} McCaffrey, D. F., Lockwood, J. R., Koretz, D., Louis, T. A., \& Hamilton, L. (2004). Models for value-added modeling of teacher effects. Journal of educational and behavioral statistics, 29(1), 67-101.
    \bibitem{Pennycook} Pennycook, G., \& Rand, D. G. (2018). Who falls for fake news? The roles of bullshit receptivity, overclaiming, familiarity, and analytic thinking.
    \bibitem{Pfeifer} Pfeifer, A., \& Lugrin, B. (2018, June). Female Robots as Role-Models?-The Influence of Robot Gender and Learning Materials on Learning Success. In International Conference on Artificial Intelligence in Education (pp. 276-280). Springer, Cham.
    \bibitem{Rashkin} Rashkin, H., Choi, E., Jang, J. Y., Volkova, S., \& Choi, Y. (2017). Truth of varying shades: Analyzing language in fake news and political fact-checking. In Proceedings of the 2017 Conference on Empirical Methods in Natural Language Processing (pp. 2931-2937).
    \bibitem{Siles} Silles, M. A. (2009). The causal effect of education on health: Evidence from the United Kingdom. Economics of Education review, 28(1), 122-128.
    \bibitem{Smith} Smith, V. C., Lange, A., \& Huston, D. R. (2012). Predictive modeling to forecast student outcomes and drive effective interventions in online community college courses. Journal of Asynchronous Learning Networks, 16(3), 51-61.
    \bibitem{Tang} Tang, C., Ouyang, Y., Rong, W., Zhang, J., \& Xiong, Z. (2018, June). Time Series Model for Predicting Dropout in Massive Open Online Courses. In International Conference on Artificial Intelligence in Education (pp. 353-357). Springer, Cham.
    \bibitem{Wang} Wang, W. Y. (2017). ``Liar, liar pants on fire": A new benchmark dataset for fake news detection. arXiv preprint arXiv:1705.00648.
  \end{thebibliography}
\end{document}